%\documentclass[tikz,border=10pt]{standalone}
%\usepackage{tikz}
%\usetikzlibrary{arrows.meta, positioning}

%\begin{document}

\begin{tikzpicture}
    [  block/.style={draw, fill=white, rectangle, minimum            height=2em, minimum width=3em},
    block2/.style = {draw, fill=white, rectangle, minimum            height=4em, minimum width=6em},
    sum/.style={draw, fill=white, circle, minimum size=2.5em, inner sep=0pt},
       lcircle/.style={draw, fill=white, circle, minimum size= 0.5em, inner sep=0pt},
       connector/.style={-Latex, thick},
       node distance=2cm and 2cm]

\node[lcircle] (input1) {};
\node[lcircle, below=2cm of input1] (input2) {};


\coordinate[below=1cm of input1] (mid);
\node[block2, right=2.5cm of mid] (prod) {$\prod$};
\coordinate[left=0cm of prod] (stutze);

\coordinate[above=0.3cm of stutze] (mult_int1);
\coordinate[below=0.3cm of stutze] (mult_int2);

\coordinate[left=1cm of mult_int1] (hilfe1);
\coordinate[left=1cm of mult_int2] (hilfe2);

\node[lcircle, right=of prod] (output) {};

\draw (input1) -| (hilfe1) node[midway,above, xshift=-3mm] {$V_x= X \cdot sin(\omega t)$};
\draw[connector] (hilfe1) -- (mult_int1);
\draw (input2) -| (hilfe2) node[midway, below, yshift=-0mm] {$V_y= Y \cdot sin(\omega t + \phi)$};
\draw[connector] (hilfe2) -- (mult_int2);

\draw[connector] (prod) -- (output) node[near end, above] {$V_o$};
    
\end{tikzpicture}

%\end{document}