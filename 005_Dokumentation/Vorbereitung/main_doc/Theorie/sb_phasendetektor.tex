%\documentclass[varwidth=true, border=10pt, crop=true]{standalone}
%\usepackage{tikz}
%\usepackage{circuitikz}
%\usetikzlibrary{positioning,arrows.meta}

%\begin{document}


 \begin{circuitikz}[european, block2/.style = {draw, fill=white, rectangle, minimum height=3em, minimum width=4.5em}, lcircle/.style={draw, fill=white, circle, minimum size= 0.25em, inner sep=0pt}, node distance=2cm and 2cm]


% === Multiplikator ===
\node[lcircle] (input1) {};
\node[lcircle, below=2cm of input1] (input2) {};


\coordinate[below=1cm of input1] (mid);
\node[block2] (prod) [right=2cm of mid] {$\prod$};
\coordinate[left=0cm of prod] (stutze);

\coordinate[above=0.3cm of stutze] (mult_int1);
\coordinate[below=0.3cm of stutze] (mult_int2);

\coordinate[left=1cm of mult_int1] (hilfe1);
\coordinate[left=1cm of mult_int2] (hilfe2);

\draw (input1) -| (hilfe1) node[near start,above, xshift=-3mm] {$V_{HP}$};
\draw (hilfe1) -- (mult_int1);
\draw (input2) -| (hilfe2) node[near start, below, xshift=-3mm] {$V_{in}$};
\draw (hilfe2) -- (mult_int2);




% === OP === 

\coordinate[right=0cm of prod] (start);
\coordinate[right=2.5cm of start] (endr);
\draw (start) to[R=R, -*] (endr);
\node[op amp, right=0cm of endr, anchor=-] (opamp) {};

\draw (opamp.+) -- ++ (0,-0.86) node[ground] {};

\coordinate (vout) at (opamp.out);
\coordinate[above=1.5cm of endr] (helpC);
\coordinate[above=2cm of vout] (startr2);
\draw (vout) -- (startr2) to[C, l=$C$] (helpC) -- (endr);

\coordinate[right=2.5cm of startr2] (endr2);
\draw (startr2) to[R=R,*-*] (endr2);
\coordinate[right=0.75cm of endr2] (Vc);
\draw (endr2) to[short,-o]  (Vc) node[above] {$V_c$};


\coordinate[below=2cm of endr2] (endr3);
\draw (endr2) to [R=R] (endr3);
\coordinate[below=1.35cm of endr3] (gnd2);
\draw (endr3) to[V=$V_H$] (gnd2);
\draw (gnd2) node[ground] {};


\end{circuitikz}


%\end{document}
