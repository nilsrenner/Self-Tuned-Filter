%\documentclass[../main_config.tex]{subfiles}
%\begin{document}

\chapter{Tools}
Im Rahmen dieser Bachelorarbeit kommen viele verschiedene Softwerkzeuge für Simulation, Messung, Schaltungsdesign und Darstellung von Messergebnissen zum Einsatz.\par
\medskip
Zu den wichtigsten Tools zählt die ECAD-Software KiCad 9. Dieses Open-Source-Programm wird unter anderem vom CERN und einer internationalen Entwicklergemeinschaft weiterentwickelt. Es umfasst eine umfangreiche Komponentenbibliothek, einen integrierten Schaltplan- und PCB-Editor, 3D-Visualisierung, zahlreiche Exportformate sowie einen eingebetteten ngspice-Simulator für die Analyse analoger Schaltungen.\par
\medskip
KiCad wird in dieser Thesis hauptsächlich für den Schaltplanentwurf, den Platinenentwurf sowie für die Simulation der Gesamt- und Teilsysteme verwendet. Der integrierte ngspice-Simulator bietet alle Analysetypen, die auch in LTspice verfügbar sind. Allerdings kann LTspice bei komplexeren Bauteilen robuster funktionieren. Daher wird für die Gesamtsimulation in Einzelfällen auf LTspice zurückgegriffen.\par
\medskip 

%!!!ACHTUNG!!! Das stimmt nicht; ngspice, der SPICE-Kern in KiCAD hat alles, wasLTspice auch hat. Schauen Sie im Handbuch von ngspice nach.

Die Datenaufnahme der realen Messwerte erfolgt mithilfe des RedPitaya STEMlab. Dies ist ein in Europa entwickeltes, vielseitiges Messgerät, das unter anderem die Funktion eines Oszilloskops, Signalgenerators und Spektrumanalysators in sich vereint.
%Es basiert auf einer (Open-Source - ich denke nicht) FPGA Entwicklungsplattform. 
Nach Abschluss einer Messung können die Daten als \text{.csv}-Datei exportiert und anschließend für die weitere Auswertung verwendet werden.\par
\medskip
Zur visuellen Aufbereitung und Analyse der Messergebnisse wird Python 3.12 in der Open-Source Entwicklungsumgebung Spyder eingesetzt. Die Funktionalität von Python ist für diese Zwecke in der Regel ausreichend, sodass auf MATLAB/Simulink nur im Ausnahmefall zurückgegriffen werden muss.\par
\medskip
Die Programmierung des Mikrocontrollers erfolgt im Editor Thonny, einer kostenlosen Open-Source Plattform für Python. Alternativ kann auch auf die Arduino IDE in der Programmiersprache C ausgewichen werden.\par
\medskip
Die Dokumentation dieser Arbeit erfolgt in \LaTeX\. Die enthaltenen Blockschaltbilder, elektronischen Schaltpläne und Plots zur Visualisierung der Ergebnisse werden
ebenfalls in \LaTeX\ TikZ/PGF realisiert. DAs soll schlussendlich für ein einheitlicheres Dokument sorgen. \par
\medskip
Zur Versionsverwaltung und Datensicherung werden Git und GitHub eingesetzt.

%\end{document}


