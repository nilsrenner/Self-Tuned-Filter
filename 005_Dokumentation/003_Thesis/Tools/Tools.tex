\documentclass[../main_config.tex]{subfiles}
\begin{document}

\chapter{Tools}
Im Rahmen dieser Bachelorarbeit kommen viele verschiedene Softwerkzeuge für Simulation, Messung, Schaltungsdesign und Darstellung von Messergebnissen zum Einsatz.\par
\medskip
Zu den wichtigsten Tools zählt die ECAD-Software KiCad 9. Dieses Open-Source-Programm wird unter anderem vom CERN und einer internationalen Entwicklergemeinschaft weiterentwickelt. Es umfasst eine umfangreiche Komponentenbibliothek, einen integrierten Schaltplan- und PCB-Editor, 3D-Visualisierung, zahlreiche Exportformate sowie einen eingebetteten SPICE-Simulator für die Analyse analoger Schaltungen.\par
\medskip
KiCad wird in dieser Thesis hauptsächlich für den Schaltplanentwurf, den Platinenentwurf sowie für die Simulation der Gesamt- und Teilsysteme verwendet. (Da im SPICE-Simulator von KiCad nicht alle Analysetypen zur Verfügung stehen, kann es erforderlich sein, für die Gesammtsimulation auf LTspice auszuweichen.)\par
\medskip
Die Datenaufnahme der realen Messwerte erfolgt mithilfe eines RedPitaya. Dies ist ein (in Europa entwickeltes,) vielseitiges Messgerät, das unter anderem die Funktion eines Oszilloskops, Signalgenerators und Spektrumanalysators in sich vereint. Es basiert auf einer Open-Source FPGA Entwicklungsplattform. Nach Abschluss einer Messung exportiert der RedPitaya die Messdaten als CSV-Datei, welche anschließend zur weiteren Auswertung genutzt wird.\par
\medskip
Zur visuellen Aufbereitung und Analyse der Messergebnisse wird Python 3.12 in der Open-Source Entwicklungsumgebung Spyder eingesetzt. Darüber hinaus wird Python zur systemtheoretischen Analyse des Gesammtsystems verwendet. (Sollte sich dabei eine unzureichende Funktionalität ergeben, kann alternativ auf MATLAB/Simulink zurückgegriffen werden.)\par
\medskip
Die Programmierung des Microcontrollers erfolgt in der Entwicklungsumgebung Thonny, einer kostenlosen Open-Source Plattform für Python. (Alternativ kann auch die Arduino IDE in der Programmiersprache C verwendet werden.)\par
\medskip
Die Dokumentation der Thesis erfolgt in LaTex. Die enthaltenen Schaubilder und elektronischen Schaltpläne werden ebenfalls in LaTex realisiert. \par
\medskip
Zur Versionsverwaltung und Datensicherung werden Git und GitHub eingesetzt.



\end{document}


