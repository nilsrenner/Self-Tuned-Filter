\documentclass[a4paper,12pt]{book}

% Sprach- und Zeichencodierung
\usepackage[T1]{fontenc}       % korrekte Darstellung von Umlauten
\usepackage[utf8]{inputenc}    % UTF-8 Eingabekodierung
\usepackage[german]{babel}    % deutsche Spracheinstellungen
\usepackage{microtype}

% Mathematik
\usepackage{amsmath, amssymb, mathtools} % erweiterte Matheumgebungen
\usepackage{siunitx}                     % SI-Einheiten und Zahlenformatierung
\sisetup{per-mode=symbol}

% Layout und Formatierung
\usepackage{geometry}          % Seitenränder einstellen
\usepackage{graphicx}          % Grafiken einbinden

\usepackage{float}             % präzise Positionierung von Bildern
\usepackage{fancyhdr}          % Kopf- und Fußzeilen
\usepackage{titlesec}          % Überschriftenformatierung
\usepackage{tocloft}           % Inhaltsverzeichnis anpassen
\usepackage{makecell}          % Zeilenumbrüche in Tabellenzellen
\usepackage{array}             % erweiterte Tabellenspalten
\usepackage{booktabs,tabularx} % schöne Tabellen
\usepackage{longtable}         % Tabellen über mehrere Seiten
\usepackage{caption}           % bessere Bild-/Tabellenunterschriften
\usepackage{subcaption}        % Subfigures
\usepackage{enumitem}          % flexible Listen
\usepackage{subfiles}          % Kapitel einzeln kompilierbar
\usepackage{tikz}
\usepackage{circuitikz}
\usetikzlibrary{arrows.meta, positioning}

\usepackage{pgfplots}
\pgfplotsset{compat=1.18}
%\newcommand{\datapath}{../../003_Messdaten}
\usepgfplotslibrary{groupplots}

% Literatur
\usepackage{csquotes}                           % saubere Anführungszeichen
\usepackage[backend=biber, style=ieee]{biblatex} % IEEE-Stil mit Biber
\addbibresource{Literaturverzeichnis.bib}       % Literaturdatenbank einbinden
\usepackage[acronym]{glossaries-extra}
\setabbreviationstyle[acronym]{long-short}
\newacronym{pll}{PLL}{Phase-Locked Loop}%

%\newacronym{digipot}{DP}{Digitalpotentiometer}

\newacronym{opv}{OPV}{Operationsverstärker}%fertig

\newacronym{pd}{PD}{Phasendetektor}%

\newacronym{vco}{VCO}{Voltage-Controlled Oscillator}%

\newacronym{vcf}{VCF}{Voltage-Controlled Filter}%

\newacronym{pcb}{PCB}{Printed Circuit Board}%

\newacronym{mcu}{MCU}{Micro Controller Unit}%

\newacronym{gpio}{GPIO}{General Purpose Input Output}

\newacronym{spi}{SPI}{Serial Peripheral Interface}

\newacronym{adc}{ADC}{Analog Digital Converter}%


\newacronym{ui}{UI}{User Interface}

\newacronym{wlan}{WLAN}{Wireless Local Area Network}

\newacronym{eeprom}{EEPROM}{Electrically Erasable Programmable Read-Only Memory}

\newacronym{http}{HTTP}{Hypertext Transfer Protocol}

\newacronym{html}{HTML}{Hypertext Markup Language}

\newacronym{ip}{IP}{Internet Protocol}

\newacronym{ram}{RAM}{Random Access Memory}

\newacronym{gbw}{GBW}{Gain Bandwidth Product}

\newacronym{sf}{SF}{Scale Factor}

\makeglossaries

% Zusatzfunktionen
\usepackage{lastpage}    % für "Seite X von Y"
\usepackage{xcolor}      % Farben (für Listings etc.)
\usepackage{listings}    % Codeblöcke
\usepackage[hidelinks]{hyperref} % Hyperlinks (immer zuletzt)


% Einstellungen für Code
\lstdefinestyle{python}{
    language=Python,
    basicstyle=\ttfamily\footnotesize,
    keywordstyle=\color{blue},
    commentstyle=\color{gray},
    stringstyle=\color{red},
    numbers=left,
    numberstyle=\tiny\color{gray},
    frame=single,
    breaklines=true,
}


% Seitenlayout
\geometry{
  top=2cm,
  bottom=3cm,
  left=2.5cm,
  right=2.5cm
}
\setlength\parindent{0pt} % kein Absatzeinzug


% Kopf- und Fußzeilen
\setlength\headheight{44pt}%26
\setlength\headsep{10pt}%35
\addtolength{\topmargin}{-12pt}

\pagestyle{fancy}
\fancyhf{}
\lhead{\raisebox{-0.2cm}{\includegraphics[height=1cm]{Bilder/hsb_logo2.png}}}
\chead{Bachelorthesis\par Nils Renner}
\rhead{\raisebox{-0.2cm}{\includegraphics[height=1.3cm]{Bilder/hsb_logo1.png}}}
\cfoot{Seite \thepage\ von \pageref{LastPage}}


% Dokumentbeginn
\begin{document}

\frontmatter
\pagenumbering{roman}
\setcounter{page}{1}

% Titelseite
\begin{titlepage}
    \begin{center}
    \includegraphics[width=0.6\textwidth]{Bilder/Logo_HSB_Hochschule_Bremen.png}\\[1cm]
    \vspace{0.5cm}
    {\huge\bfseries Implementierung eines selbsteinstellenden Filters auf Basis eines spannungsgesteuerten Biquad-Filters\par}
    \vspace{1.5cm}
    {\large\bfseries Implementation of a Self-Tuned Filter Based on a Voltage-Controlled Biquad Filter\par}
    \end{center}
    \vspace{1.5cm}
    {\Large\bfseries Bachelor Thesis\par}
    Presented for Attainment of the Academic Degree of \par
    Bachelor of Engineering at the City University of Applied Sciences Bremen\par
    \vspace{2.5cm}

    \begin{tabular}{l l}
            \textbf{Autor:} & Nils Renner, Wulfhoopstraße 34a, 28201 Bremen \\
            \textbf{Matr.-Nr.:} & 5197659 \\
            \textbf{Abgabetermin:} & 2. März, 2026 \\ \\ \\
            \textbf{Prüfer:} & Prof. Dr.-Ing. Mirco Meiners,\\ & \small{Hochschule Bremen,}\\ & \small{Concept Engineering SoCs and Design}\\ \\
            \textbf{} & Prof. Dr. Sören Peik,\\ & \small{Hochschule Bremen,}\\ & \small{Microwave Technology and Satellite Communications} \\
    \end{tabular}

    \vfill
\end{titlepage}

\newpage
\null
\thispagestyle{empty} 


% Erklärung zur selbstständigen Arbeit
\clearpage{\pagestyle{empty}\cleardoublepage}
\chapter*{Erklärung zur selbstständigen Arbeit}
\addcontentsline{toc}{chapter}{Erklärung zur selbstständigen Arbeit}
%\href{https://www.hs-bremen.de/assets/hsb/de/Dokumente/ZLL/MMCC/KI-Eigenst\%C3\%A4ndigkeitserkl\%C3\%A4rung\_Nov\_2024\_Form.pdf}{https://www.hs-bremen.de/assets/hsb/de/Dokumente/ZLL/MMCC/KI-Eigenständigkeitserklärung\_Nov\_2024\_Form.pdf}





% Acknowledgements / Danksagung
\clearpage{\pagestyle{empty}\cleardoublepage}
\chapter*{Acknowledgements}
\addcontentsline{toc}{chapter}{Danksagung}

Danksagung



% Abstract
\clearpage{\pagestyle{empty}\cleardoublepage}
\chapter*{Abstract}
\addcontentsline{toc}{chapter}{Abstract}

englisches Abstract



% Inhaltsverzeichnis
\clearpage{\pagestyle{empty}\cleardoublepage}
\tableofcontents


% Abbildungsverzeichnis
\clearpage{\pagestyle{empty}\cleardoublepage}
\listoffigures
\addcontentsline{toc}{chapter}{Abbildungsverzeichnis}



% Tabellenverzeichnis 
\clearpage{\pagestyle{empty}\cleardoublepage}
\listoftables
\addcontentsline{toc}{chapter}{Tabellenverzeichnis}



% Abkürzungsverzeichnis
\clearpage{\pagestyle{empty}\cleardoublepage}
\printglossary[type=\acronymtype, title=Abkürzungsverzeichnis, nonumberlist]
%\addcontentsline{toc}{chapter}{Abkürzungsverzeichnis}



\mainmatter
% Haupttext
% \newpage  
% \pagenumbering{arabic}  
% \setcounter{page}{1}

\subfile{Einleitung/Einleitung.tex}

% \newpage
\subfile{Tools/Tools.tex}

% \newpage
\subfile{Theorie/Theorie_alt.tex}
% \newpage
\subfile{Theorie/Theorie_neu.tex}

% \newpage
\subfile{Simulation/Simulation.tex}

% \newpage
\subfile{systemdesign/systemdesign.tex}

% \newpage
\subfile{Messung/Messung.tex}

% \newpage
\subfile{Auswertung/Auswertung.tex}

% \newpage
\subfile{Fazit/Fazit.tex}

\backmatter
\appendix
% \newpage
\printbibliography
% \newpage

\end{document}
