\documentclass[a4paper,12pt]{book}

% Sprach- und Zeichencodierung
\usepackage[T1]{fontenc}       % korrekte Darstellung von Umlauten
\usepackage[utf8]{inputenc}    % UTF-8 Eingabekodierung
\usepackage[german]{babel}    % deutsche Spracheinstellungen
\usepackage{microtype}

% Mathematik
\usepackage{amsmath, amssymb, mathtools} % erweiterte Matheumgebungen
\usepackage{siunitx}                     % SI-Einheiten und Zahlenformatierung


% Layout und Formatierung
\usepackage{geometry}          % Seitenränder einstellen
\usepackage{graphicx}          % Grafiken einbinden

\usepackage{float}             % präzise Positionierung von Bildern
\usepackage{fancyhdr}          % Kopf- und Fußzeilen
\usepackage{titlesec}          % Überschriftenformatierung
\usepackage{tocloft}           % Inhaltsverzeichnis anpassen
\usepackage{makecell}          % Zeilenumbrüche in Tabellenzellen
\usepackage{array}             % erweiterte Tabellenspalten
\usepackage{booktabs,tabularx} % schöne Tabellen
\usepackage{longtable}         % Tabellen über mehrere Seiten
\usepackage{caption}           % bessere Bild-/Tabellenunterschriften
\usepackage{subcaption}        % Subfigures
\usepackage{enumitem}          % flexible Listen
\usepackage{subfiles}          % Kapitel einzeln kompilierbar
\usepackage{tikz}
\usepackage{circuitikz}
\usetikzlibrary{arrows.meta, positioning}

% Literatur
\usepackage{csquotes}                           % saubere Anführungszeichen
\usepackage[backend=biber, style=ieee]{biblatex} % IEEE-Stil mit Biber
\addbibresource{Literaturverzeichnis.bib}       % Literaturdatenbank einbinden


% Zusatzfunktionen
\usepackage{lastpage}    % für "Seite X von Y"
\usepackage{xcolor}      % Farben (für Listings etc.)
\usepackage{listings}    % Codeblöcke
\usepackage[hidelinks]{hyperref} % Hyperlinks (immer zuletzt)


% Einstellungen für Code
\lstdefinestyle{python}{
    language=Python,
    basicstyle=\ttfamily\footnotesize,
    keywordstyle=\color{blue},
    commentstyle=\color{gray},
    stringstyle=\color{red},
    numbers=left,
    numberstyle=\tiny\color{gray},
    frame=single,
    breaklines=true,
}


% Seitenlayout
\geometry{
  top=2cm,
  bottom=3cm,
  left=2.5cm,
  right=2.5cm
}
\setlength\parindent{0pt} % kein Absatzeinzug


% Kopf- und Fußzeilen
\setlength\headheight{44pt}%26
\setlength\headsep{10pt}%35
\addtolength{\topmargin}{-12pt}

\pagestyle{fancy}
\fancyhf{}
\lhead{\raisebox{-0.2cm}{\includegraphics[height=1cm]{Bilder/hsb_logo2.png}}}
\chead{Bachelorthesis\par Nils Renner}
\rhead{\raisebox{-0.2cm}{\includegraphics[height=1.5cm]{Bilder/hsb_logo1.png}}}
\cfoot{Seite \thepage\ von \pageref{LastPage}}


% Dokumentbeginn
\begin{document}

% Titelseite
\begin{titlepage}
    \centering
    \includegraphics[width=0.6\textwidth]{Bilder/Logo_HSB_Hochschule_Bremen.png}\\[1cm]
    {\scshape\LARGE Hochschule Bremen\\}
    {\scshape\Large Fakultät 4: Elektrotechnik und Informatik\\[1.5cm]}
    {\huge\bfseries Implementierung eines selbsteinstellenden Filters auf Basis eines spannungsgesteuerten Biquad-Filters\\[0.5cm]}
    {\Large\bfseries Nils Renner (5197659)\\[2cm]}
    {\Large\bfseries Prüfer\\[0.5cm]}
    \begin{tabular}{l}
      Prof. Dr.-Ing. Mirco Meiners \\
      Prof. Dr. Sören Peik
    \end{tabular} \\[2cm]
    \vfill
\end{titlepage}


% Inhaltsverzeichnis
\newpage
\pagenumbering{gobble} % keine Seitenzahl auf Inhaltsverzeichnis
\tableofcontents


% Haupttext
\newpage  
\pagenumbering{arabic}  
\setcounter{page}{1}



\subfile{Einleitung/Einleitung.tex}

\newpage
\subfile{Tools/Tools.tex}


\newpage
\subfile{Theorie/Theorie_alt.tex}
\newpage
\subfile{Theorie/Theorie_neu.tex}

\newpage
\subfile{Simulation/Simulation.tex}

\newpage
\subfile{systemdesign/systemdesign.tex}

\newpage
\subfile{Messung/Messung.tex}

\newpage
\subfile{Auswertung/Auswertung.tex}

\newpage
\subfile{Fazit/Fazit.tex}

\newpage
\printbibliography
\newpage

\listoffigures

\end{document}
