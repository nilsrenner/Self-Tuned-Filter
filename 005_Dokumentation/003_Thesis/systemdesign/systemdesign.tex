\documentclass[../main_config.tex]{subfiles}
\begin{document}


\chapter{Schaltungsentwurf/ Design des Systems}

Anfangs wird die Schaltung ohne Ansteuereung durch den Mikrocontroller und ohne Perepherie auf dem ASLK-PRO-Board aufgebaut. Auffällig ist dabei, dass am SF-Pin des Multiplizierers statt der im Datenblatt angegebenen \SI{10}{\volt} nur etwa \SI{8.78}{\volt} anliegen. Grund dafür ist, dass die Versorgungsspannung des Multiplizierers auf dem ASLK-PRO-Board nur bei \SI{\pm10}{\volt} liegt, anstatt der im Datenblatt vorgeschlagenden \SI{\pm15}{\volt}. Dieser Unterschied sollte die Funktion des Multiplizierers zwar nicht beeinträchtigen, für eine bessere Verständlichkeit der Schaltung wären die \SI{\pm15}{\volt} aber hilfreich. Der verwendete \gls{opv} TL082B ist laut Datenblatt bis zu \SI{\pm20}{\volt} verwendbar, sodass die erste Platine mit \SI{\pm15}{\volt} Versorgungsspannung geplant wird. \par
\medskip





\section{Design des Schaltplans}
Die im VCF verwendeten Bauteile wurden größtenteils vom Aufbau des letzen Semester bzw. dem ALSK-PRO Manual \cite{Lab_Kit_PRO} übernommen. Bei der Wahl des \gls{opv} soll in Version \num{1} der Platine der gleiche OpAmp verwendtet werden wie schon zuvor. Dabei wird darauf geachtet, den für Filteranwendungen etwas besseren TL082B zu verwenden. Dieser baut auf der gleichen Architektur auf, bietet aber leicht verbesserte Werte im Bereich der Input Offset Voltage und Input Offset Drift. Dadurch werden besonders im Bereich der Integration fehler minimiert, da diese Version etwas präzieser arbeitet. Das GBW und die Slew-Rate werden durch die Wahl der Version nicht beeinflusst.\par
\medskip
Für die Digitalpotentiometer fällt die Wahl auf das von Herrn Ziemann vorgeschlagene MCP4261. Bei der weiteren Auswahl werden die mögliche Schrittanzahl sowie der Speichertyp berücksichtigt. \textbf{was Speichertyp berücksichtigt?} Da zur Einstellung des Güte- und Verstärkungsfaktors sowie der Mittenfrequenz des Filters insgesammt vier Potentiometer gebraucht werden, wird ein Modul gewählt, bei dem sich zwei Potis un einem Gehäuse befinden. Der Baustein ist zudem als Potentiometer und Rheostat erhältlich, wobei das Rheostat die gewollten strombegrenzenden Eigenschaften besitzt, während das Poti eine Spannung herausgibt. Trotzdem wird inn diesem Fall die Potentiometerversion verwendet, da diese leichter in PDIP-Gehäusen erhäldlich ist und sich durch das Offenlassen eines Pins (nicht der Abgriff) als verstellbarer Widerstand einsetzen lässt. \par
\medskip



Die Ein- und Ausgänge der Potentiometer werden jeweils mit Jumpern versehen, damit der eingestellte Widerstandswert schnell und ohne Beeinflussung durch die restliche Schaltung gemessen werden kann. Die Multiplizierer werden an den vorgesehenden Stellen in den Biquad eingesetzt, wobei der Scale-Faktor-Pin so verschaltet wird, dass dessen Potential über einen Spindeltrimmer einstellbar ist und sich die Proportionalitätskonstante des Multiplizierers anpassen lässt.

%Laut Datenblatt soll der SF-Pin des Multiplizierers auf \SI{-10}{\volt} lasergetrimmt sein. Auf der Übungs-Steckplatine wird dieser Wert nie erreicht, da die Spannungsversorgung dieser bei \SI{\pm 10}{\volt} liegt. In Version 1 der Platine wird als Engangspannung \SI{\pm 15}{\volt} gewählt, damit die Spannuung an SF als \SI{10}{\volt} angenommen werden kann.
\par
\medskip
Zum Zeitpunkt der Schaltplanerstellung ist die genaue Funktion der Hilfsspannung $V_H$ noch nicht genau bekannt. Damit später trotzdem der optimale Wert einstellbar ist wird eine Schaltung designed, die einen Spannungswert zwischen \SI{\pm 15}{\volt} ausgeben kann. \par
\medskip
Um die Signale der verschiedenen Filtertypen besser mit geeeigneten Messinstrumenten wie Oszilloskop oder Spektrumanalysator aufnehmen zu können, werden BNC-Amschlüsse auf der Platiene angebracht, da diese Messgeräte meist BNC-Eingänge besitzen.\par
\medskip
Auf der Platine übernimmt der Mikrocontroller die Ansteuerung der Digitalpltentiometer. Später soll dieser zudem eine Weboberfläche zur Bedienung des gesammten Filters bereitstellen. Hierfür wird ein Raspberry Pico 2 W verwendet, dessen Mikrokontroller RP2350 in Europa entworfen wurde. Er ist eine Weiterentwicklung des RP2040 und sollte für diese Aufgabe stark genug sein. Zusätzlich befindet sich auf dem Pico 2 Modul der WLAN Baustein CYW43439 von Infinion, der die drahtlose Kommunikation (WLAN und Bluetooth) zwischen Eingabegerät und Filter ermöglicht. \par
\medskip
Sowohl die Digitalpotentiometer als auch der Mikrocontroller benötigen Versorgungsspannungen im Bereich von etwa \SIrange{1.8}{5.5}{\volt}. Da auch Busse und allgemein Pins des µC mit \SI{3.3}{\volt} betrieben werden, wird eine Versorgungsspannung \SI{3.3}{\volt} integriert. Um dieses Potential zu erreichen ohne sehr hohe Verluste zu generieren, soll ein Buck-Converter die \SI{+15}{\volt} auf ein Niveau von \SI{3.3}{\volt} absenken. 
Die zugehörigen externen Bauteile des Buck-Converters werden wie im Datenblatt angegeben anhand des Maximalstroms, der maximalen Eingangsspannung und der Ausgangsspannung dimensioniert. \par
\medskip
Ein weiterer \gls{opv} wird zur Frequenzbestimmung des Eingangssignals genutzt. Er wird als invertierender Schmitt‑Trigger beschaltet, um das eingehende Sinussignal in ein Rechtecksignal zu überführen. Dieses Rechtecksignal soll dann über die Nulldurchgangsmethode im Mikrocontroller die Eingangsfrequenz bestimmen. \par
\medskip

Zuletzt wird überprüft, ob alle relevanten Signalpfade mit Testpunkten versehen sind, um während der Messungen möglichst viele interne Signale des Self-Tuned~Filters aufnehmen zu können. Das soll später dabei Helfen potentielle Fehler schneller zu identifizieren. Zusätzlich werden Mounting-Holes gesetzt, damit die Platine sicher fixiert werden kann und der Zugriff auf die Edge‑Mount‑BNC‑Steckverbinder problemlos funktioniert. Insgesammt wird der Schaltplan so gestaltet, dass er möglichst übersichtlich und gut nachvollziehbar bleibt.






\section{Design der Platine}

Beim Platinendesign muss zuerst entchieden werden, wie viele Lagen die Platine haben soll. An sich kann jeder beliebige Schaltplan auf einer zweilagigen Platine umgesetzt werden. Dies geht allerdings auf Kosten der Platinengröße und kann Störeinflusse begünstigen. Um diese beiden Aspekte zu minimieren, wird ein vierlagiges Layout gewählt, da dies das Routing erheblich einfacher macht und die Mehrkosten überschaubar sind. \par
\medskip
Ein wesendlicher Vorteil von mehrlagigen Platinen besteht darin, dass Bauteile deutlich dichter an einander plaziert werden können, ohne dass sich dazwischenliegende Leiterbahnen gegenseitig behindern. Dieser Vorteil wirkt sich noch stärker auf PCBs mit SMD-Komponenten aus, da der Footprint nur auf der ersten Kupferlage zu erkennen ist und die anderen Lagen nicht aktiv beeinflusst. Trotzdem wurde in der ersten Iteration der Platine absichtlich auf SMD-Komponenten verzichtet, da THT-Elemente im Umgang einfacher sind und bei Anpassungen leichter auszutauschen sind.


Platzbedingte Vorteile der Vielagigen platine bestehen vor allen dingen darin, dass die Verwendeten Bauteile näher an einander plaziert werden können, ohne dass sich die dazwischenligenden Leiterbahnen behindern. Dieser Vorteil wirkt sich noch stärker auf PCBs mit SMD-Komponenten aus, auf diese wurde allerdings in der ersten Iteration absichtlich verzichtet, da der Umgang mit THT-Bauteilen einfacher ist und diese bei eventullen anpassungen leichter auszutauschen sind. 

Die allgemeinen Aufgaben der vier Lagen werden wie folgt zugeordnet:

\begin{enumerate}
    \item Layer 1: Signalrouting \par
    Die oberste Lage der Platine dient hauptsächlich dem Signalrouting. Hierbei sollen so weit es geht alle an der Filterung beteiligten Signalpfade über diese Ebene geleitet werden. Auch wenn der in dieser Abeit betrachtete Frequenzbereich noch recht unanfällig für Störungen ist, wird darauf geachtet, diese Störeinflüsse so gering wie möglich zu halten.
    \item Layer 2: Groundlayer \par
    Direkt darunter befindet sich eine durchgehende Massefläche. Diese Schicht bleibt ununterbrochen, sodass das Ground-Potential über die THT-Pins beziehungsweise Vias an jeder Stelle der Platine einfach erreichbar ist. Zudem kann diese Fläche potentiell störenden Bussignale abschirmen. 
    \item Layer 3 Powerlayer \par
    Diese Schicht dient ausschließlich der Spannungsversorgung der Bauteile auf der Platine. Die \SI{-15}{\volt} sowie die \SI{3.3}{\volt} werden dabei ganz normal geroutet, die \SI{+15}{\volt} verlaufen hingegen großflächig über die Ebene, da dieses Potential sowohl die \gls{opv} betreibt, als auch die \SI{3.3}{\volt} von diesem Signal aus abgehen. Dies ist die einzige Ebene in der die Zone nicht auf Ground gelegt wird.
    \item Layer 4 Signallayer \par
    Die unterste Schicht dient als Ausweichmöglichkeit für sich kreuzende Signale. An sonsten soll sie hauptsächlich für das Routing von Bussignalen verwendet werden. In Version 1 wurde darauf noch nicht so genau geachtet, in der (\textbf{Endversion!!}) hingegen schon. 
\end{enumerate}

Bei der Plazierung der Komponenten wird darauf geachtet, das Komunikationsmodul des Mikrokontrollers ganz an den Rand der Platine zusetzen, um die Abschirmung der Antenne durch die Kupferflächen der Platine zu vermeiden. Bei der Platizerung des Buck-Converter und dazgehörigen weiteren Komponenten wird darauf geachtet, dass diese wie im Datenblatt beschrieben auf dem PCB plaziert werden. Aufgrund von Platzmangel hat sich das Layout dennoch etwas verändert.\par
\medskip

Da in der linken oberen Ecke der Platine noch etwas Platz ist, wird für einen einfachen Zugang auf die Dokumentation des Projekts ein QR-Code zum Git-Reposatorium der Arbeit hinzugefügt. 



\section{Design des Codes}

%beispiel:

%\begin{lstlisting}[style=python]
%print("Hello World")
%\end{lstlisting}

\subsection{Design der Website}


\end{document}





