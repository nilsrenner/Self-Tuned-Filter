\documentclass[../main.tex]{subfiles}
\begin{document}

\chapter{Weiterführende Theorie}
\glsresetall %da ich in diesem kapitel einmal die langform wieder haben möchte (für PLL und VCF hats vorher nicht geklappt)


In der bisherigen Analyse des Biquad-Filters stand das Amplitudenverhalten im Vordergrund, während die Phase nur eine untergeordnete Rolle bei Aufnahme der Filtercharakteristik spielte. Für die folgende Analyse des selbsteinstellenden Filters ist die Phase jedoch von zentraler Bedeutung, da sie zur automatischen Einstellung der Filterparameter verwendet wird. Ziel dieses Kapitels ist, die theoretischen Grundlagen und Funktionsprinzipien des selbsteinstellenden (Self-Tuned) Filters zu erarbeiten und dessen Funktionsprinzipien zu analysieren. (Dafür wird zunächst die Funktion der einzelnen Bausteine analysiert und mittels Simulationen verifiziert. Anschließend werden die einzelnen Bausteine zusammengefügt und das Gesamtsystem betrachtet.)\par
\medskip
Da der Self-Tuned Filter in seiner Struktur und Funktion starke Parallelen zu einem \gls{pll} aufweist, werden Anfangs die Grundkonzepte des \gls{pll} erläutert. Aufbauend darauf wird der analoge Multiplizierer als \gls{pd} eingeführt und dessen Verhalten sowohl analytisch als auch simulativ untersucht. Anschließend erfolgt die Betrachtung des \gls{vcf}, dessen Steuerung über die detektierte Phaseninformation die Grundlage für die automatische Anpassung der Filtergrenzfrequenz bildet.\par
\medskip
Ein weiterer Schwerpunkt dieses Kapitels liegt auf der Sensitivitätsanalyse der beteiligten Komponenten sowie auf der Untersuchung theoretischer und praktischer Grenzen des Selbstabstimmbereichs. Abschließend wird die Fähigkeit des Gesamtsystems zur Frequenzdetektion des Eingangssignals analysiert und bewertet.\par
\medskip
Da zum Thema selbstabstimmender analoger Filter nur begrenzt wissenschaftliche Literatur verfügbar ist, dient der etablierte \gls{pll} im Verlauf dieses Kapitels wiederholt als Referenzmodell zur Einordnung und Erklärung der zugrunde liegenden Funktionsprinzipien.





\section{Einführung in den Phasenregelkreis}     
Bei einem \gls{pll} handelt es sich um eine geschlossene Rückkopplungsschleife, in der die Phase eines internen Signals, wie z.B. dem Ausgang eines \gls{vco}, an die Phase eines stabilen, externen Referenzsignals angepasst wird. Sobald die Signale synchron zueinander verlaufen (locked), besitzen das interne Signal und das Referenzsignal die gleiche Frequenz. Wenn die Frequenz des Referenzsignals verändert wird, versucht die elektronische Schaltung die Synchronisation aufrecht zu erhalten bzw. wieder zu erlangen. Das Ausgangssignal des \gls{vco} kann dem eingehenden Steuersignal also über einen gewissen Frequenzbereich folgen.\cite{chengCom}\par
\medskip
Der einfache Aufbau eines \gls{pll} besteht aus einem Phasen Detektor, einem Loopfilter und einem \gls{vco}. Diese werden wie in der folgenden Abbildung \ref{fig:bsb_pll} dargestellt.\par


\begin{figure}[H]
  \centering
  \resizebox{0.8\textwidth}{!}{\begin{tikzpicture}
    [%
    block/.style={draw, fill=white, rectangle, minimum height=2em, minimum width=3em},
    block2/.style={draw, fill=white, rectangle, minimum height=4em, minimum width=6em},
    sum/.style={draw, fill=white, circle, minimum size=2.5em, inner sep=0pt},
    bcircle/.style={draw, fill=black, circle, minimum size= 0.5em, inner sep=0pt},
    lcircle/.style={draw, fill=white, circle, minimum size= 0.5em, inner sep=0pt},
    connector/.style={-Latex, thick},
    node distance=2cm and 2cm
    ]

  \node[lcircle] (input1) at (0,0) {};
\node[block2] (prod) at (3.5,-0.3) {$PD$};
\coordinate[left=0cm of prod] (stütze);

\coordinate[above=0.3cm of stütze] (mult_int1);
\coordinate[below=0.3cm of stütze] (mult_int2);
\node[lcircle, right=of prod] (mult_out) {};

\node[block2] (loopfilt) at (7,-0.3) {$Loop Filter$};
\node[block2] (VCO) at (10.5,-0.3) {$VCO$};

\node[bcircle] (loop) at (12.5,-0.3) {};
\node[lcircle] (out) at (14,-0.3) {};
\coordinate[below=2cm of loop] (loop_stütz1);
\coordinate[left= 11cm of loop_stütz1] (loop_stütz2);

\draw[connector] (input1) -- (mult_int1) node[near start,above] {$V_{in}$};
\draw[connector] (prod) -- (loopfilt) node[midway, above] {$V_{d}$};
\draw[connector] (loopfilt) -- (VCO) node[midway, above] {$V_{c}$};
\draw[connector] (VCO) -- (out) node[near end,above] {$V_{out}$};
\draw(loop) -- (loop_stütz1);
\draw(loop_stütz1) -- (loop_stütz2);
\draw[connector] (loop_stütz2) |- (mult_int2);

\end{tikzpicture}
}
  \caption{Einfacher Aufbau eines \gls{pll} \cite{razaviRF}}
  \label{fig:bsb_pll}
\end{figure}


Dabei bestimmt der \gls{pd} die Phasendifferenz zwischen dem Referenzsignal und dem Ausgangssignal des \gls{vco}. Dieses Signal wird im darauffolgenden Schleifenfilter geglättet, sodass die bei der Phasendetektion entstehenden Obertöne (hochfrequente Signalanteile) unterdrückt werden. Der anschließende \gls{vco} gibt anhand seiner Eingangsspannung eine Frequenz aus, die proportional zu seiner Eingangsspannung ist. Stimmt diese Ausgangsfrequenz nun mit der Frequenz des Referenzsignals überein ist der \gls{pll} locked.\cite{chengCom}\par
\medskip
Im Folgenden werden die ersten beiden Bausteine des \gls{pll} genauer betrachtet. Um jedoch den \gls{pd} zu verstehen, muss zunächst die Funktionsweise des analogen Multiplizierers erläutert werden.\par


\section{Analoger Multiplizierer}
\label{sec:multi}
Der zentrale Baustein des Phasendetekors ist der analoge Multiplizierer. Wie der Name bereits andeutet, bildet ein analoger Multiplizierer das Produkt aus zwei Eingangssignalen nach dem Schema: $V_{out} =V_x \cdot V_y$. 

\begin{figure}[H]
  \centering
  \resizebox{0.4\textwidth}{!}{%\documentclass[tikz,border=10pt]{standalone}
%\usepackage{tikz}
%\usetikzlibrary{arrows.meta, positioning}

%\begin{document}

\begin{tikzpicture}
    [  wcircle/.style={draw, fill=white, circle, minimum size=2.5em, inner sep=0pt},
       lcircle/.style={draw, fill=white, circle, minimum size= 0.5em, inner sep=0pt},
       connector/.style={-Latex, thick},
       node distance=2cm and 2cm]

\node[lcircle] (input1) {};
\node[wcircle, right=of input1] (gain) {$\times$};
\node[lcircle, below=of gain] (input2) {};
\node[lcircle, right=of gain] (output) {};


\draw[connector] (input1) -- (gain) node[near start,above] {$V_x$};
\draw[connector] (input2) -- (gain) node[near start, left] {$V_y$};
\draw[connector] (gain) -- (output) node[near end, above] {$V_{out}$};
    
\end{tikzpicture}

%\end{document}}
  \caption{Blockschaltbild des analogen Mulitpliziers}
  \label{fig:bsb_einfacher_multi}
\end{figure}

Wie bereits durch das bisherige Studium bekannt ist, können arithmetische Operationen wie Addition, Subtraktion und Integration über einen \gls{opv} mit entsprechender Verschaltung durchgeführt werden. Die Multiplikation zweier Signale lässt sich hingegen nicht so einfach über eine einfache analoge Schaltung realisieren. Eine mögliche Lösung dieses Problems ist der Umweg über den natürlichen Logarithmus und die Exponentialfunktion $e^x$. Über diesen Umweg kann die Multiplikation als einfache Addition durchgeführt werden.\par

\begin{equation*}
V_{out}=V_x \cdot V_y = e^{ln(V_x \cdot V_y)} = e^{ln(V_x) + ln(V_Y)}
\end{equation*}

Dieser Zusammenhang wird durch folgendes Blockschaltbild verdeutlicht:\par

\begin{figure}[H]
  \centering
  \resizebox{0.8\textwidth}{!}{%\documentclass[tikz,border=10pt]{standalone}
%\usepackage{tikz}
%\usetikzlibrary{arrows.meta, positioning}

%\begin{document}


\begin{tikzpicture}
    [  block/.style={draw, fill=white, rectangle, minimum            height=2em, minimum width=3em},
    sum/.style={draw, fill=white, circle, minimum size=2.5em, inner sep=0pt},
       lcircle/.style={draw, fill=white, circle, minimum size= 0.5em, inner sep=0pt},
       connector/.style={-Latex, thick},
       node distance=2cm and 2cm]

\node[lcircle] (input1) {};
\node[lcircle, below=2cm of input1] (input2) {};
\node[block, right=2cm of input1] (log1) {$log()$};
\node[block, right=2cm of input2] (log2) {$log()$};

\coordinate[below=1cm of input1] (mid);
%\coordinate[right=1cm of gain4] (afterW0);
\node[sum, right=4.5cm of mid] (sum) {$\sum$};

\node[block, right=1.5cm of sum] (exp) {$exp$};
\node[lcircle, right=of exp] (output) {};

\draw[connector] (input1) -- (log1) node[near start,above] {$V_x$};
\draw[connector] (input2) -- (log2) node[near start, above] {$V_y$};
\draw[connector] (log1) -| (sum);
\draw[connector] (log2) -| (sum);
\draw[connector] (sum) -- (exp);

\draw[connector] (exp) -- (output) node[near end, above] {$V_o$};
    
\end{tikzpicture}
%\end{document}}
  \caption{Funktionelle Realisierung eines Analogmultiplizierers \textbf{Quelle: Wikipedia}}
  \label{fig:bsb_multi_real}
\end{figure}

\medskip
Da der Logarithmus nicht für negative Zahlen definiert ist, können nur positive Eingangssignale multipliziert werden. Aus diesem Grund werden Multiplizierer dieses Typs auch Ein-Quadranten-Multiplizierer genannt.\par 
\medskip
In vielen Anwendungen sollen allerdings auch negative Eingangsspannungen zu einem korrekten Ergebins führen. Eine Methode um negative Eingangsspannungen verarbeiten zu können funktioniert, indem das Vorzeichen am Ein- und Ausgang des Multiplizierers umgekehrt wird. Leider ist diese Methode schaltungstechnisch aufwendig und relativ langsam, was sie für höherfrequente Anwendungen ungeeignet macht. In einer anderen Methode wird zu den Eingangsspannungen eine konstante Gleichspannung hinzuaddiert. Dadurch wird sichergestellt, dass das Potential an den Eingängen immer im positiven Bereich liegt. Die Gleichung für die Augangsspannung dieser Methode lautet dann\par

\begin{equation}
  V_{out} = \frac{(V_x+V_{xk})(V_y+V_{yk})}{E}
\end{equation}

wobei
\begin{itemize}
  \item $V_x$ und $V_y$ die Eingangssignale dartellen,
  \item $V_{xk}$ und $V_{yk}$ die konstannten Gleichspannungen sind,
  \item $E$ die Proportionalitätskonstante beschreibt, in der Praxis häufig als \SI{10}{\volt} angewendet.
\end{itemize}


Die Proportionalitätskonstante $E$ findet sich in den meisten Gleichungen zur Beschreibung des Ausgangs eines Multiplizierers. Sie sorgt dafür, dass das Ausgangssignal innerhalb des gewünschten Spannungsbereich bleibt und auch starke Verstärkungen korrekt im Pegel der Ausgangsspannung zu sehen sind. Das gewünschte Ausgangssignal $\frac{V_xV_y}{E}$ ergibt sich also aus 

\begin{equation}
  \frac{V_xV_y}{E} = V_{out}-V_x \frac{V_{yk}}{E}-V_y \frac{V_{xk}}{E}- \frac{V_{xk}V_{yk}}{E}
\end{equation}


Liegt die Eingangsspannung $V_x$ im Bereich $-E \leqq V_x \leqq +E$ kann keine negative Spannung am Eingang des Multiplizierer anliegen, wenn die konstante Spannung $V_{xk} = E$  gesetzt wird. Gleiches gilt auch für den zweiten Eingang. Bei Anwendung diese Zusammenhangs auf die bekannten Gleichungen ergibt sich für den Ausgang eines Vier-Quadranten-Multiplizierers folgende Gleichung

\begin{equation}
  V_{out} = \frac{V_xV_y}{E} = \frac{(V_x+E)(V_y+E)}{E} - V_x -V_y - E
\end{equation}

Wobei sie sich für die Umsetzung als Blockschaltbild so erweitert

\begin{equation}
  V_{out} = \frac{V_xV_y}{E} = 4 \cdot \frac{\frac{1}{2}(V_x+E) \cdot \frac{1}{2}(V_y+E)}{E} - V_x -V_y - E
\end{equation}

\begin{figure} [H]
    \centering
    \includegraphics[width=1\linewidth]{../Bilder/multi_erweiterung.png}
    \caption{Vom Einquadranten- zum Vierquadranten-Multiplizierer\cite{halbleiter_4Quad}}
    \label{fig:multi_erweiterung}
\end{figure}

\medskip
In der Realität entstehen bei der Multiplikation zweier Signale immer kleine Abweichungen und Fehler vom idealen Verhalten. Das ALSK-PRO-Manual zeigt, wie diese Abweichungen zusammengesetzt sind\par

\begin{equation*}
V_o=V_{offset}+K_x \cdot V_x + K_y \cdot V_y + K_o \cdot V_x \cdot V_y \cdot \xi
\end{equation*}

wobei
\begin{itemize}
  \item $V_{offset}$ den konstanten Offset beschreibt,
  \item $K_x V_x$, $K_y V_y$ die linearen Anteile (Störgrößen) sind, die in einem idealen Multiplizierer nicht vorkommen,
  \item $K_o V_x V_y$ den eigentlichen Multiplikationsterm darstellt,
  \item $\xi$ der Rausch- oder Restfehler ist.
\end{itemize}

Trotz möglicher Abweichungen in der Durchführung wird im Weiteren erst einmal ideal weitergerechent. Diese kleine Einführung in die Ungenauigkeiten bei der Simulation wird \textbf{vielleicht} später nochmals in der Messung aufgegriffen. 
\medskip

In dieser Bachelorarbeit wird ein MPY634 von Texas Instruments (TI) verwendet. Die allgemeine Übertragungsfunktion des MPY634 lautet wie folgt:

\begin{equation}
  V_\text{out}
  = A \left[ \frac{(X_1 - X_2)(Y_1 - Y_2)}{SF} - (Z_1 - Z_2) \right],
  \label{eq:datasheet_multi1}
\end{equation}

wobei
\begin{itemize}
  \item $A$ die offene Verstärkung (open-loop gain) des internen Verstärkers darstellt (typisch \SI{85}{\decibel}),
  \item $SF$ der Skalierungsfaktor (scale factor) ist, der ab Werk auf \SI{10}{\volt} lasergetrimmt ist, aber durch Anschluss eines Potentiometers zwischen Pin \emph{SF} und $-V_{S}$ im Bereich von \SI{3}{\volt} bis \SI{10}{\volt} einstellbar bleibt,
  \item $X$, $Y$ und $Z$ jeweils differenzielle Eingangsspannungen sind.
\end{itemize}


Die maximale Eingangsspannung sollte das \num{1,25}-fache des eingestellten Skalierungsfaktors nicht überschreiten.

\medskip
Um eine stabile, geschlossene Übertragungsfunktion zu erhalten, ist eine negative Rückkopplung erforderlich. Ohne diese würde die große Verstärkung $A$ schon bei kleinsten Abweichungen innerhalb der Klammer den Ausgang bis zum Maximalwert treiben. Wird nun $Z_1$ mit $V_{out}$ verbunden und $Z_2$ auf Masse gelegt, so ergibt sich durch Einsetzen in \eqref{eq:datasheet_multi1} die Näherung 

\begin{equation*}
  \frac{(X_1 - X_2)(Y_1 - Y_2)}{SF} - (V_\text{out} - 0) \approx 0.
\end{equation*}

Daraus folgt die geschlossene Übertragungsfunktion

\begin{equation}
  V_\text{out} =
  \frac{(X_1 - X_2)(Y_1 - Y_2)}{SF}.
  \label{eq:datasheet_multi2}
\end{equation}

Da die Analyse im ASLK-PRO-Manual immer von der Spannung $V_r$ als interne Referenz des Multiplizierers ausgeht, wird im Folgenden nur noch $V_r$ anstatt $SF$ verwendet. Beide beschreiben dieselbe Spannung, $V_r$ ist somit werksseitig auf \SI{10}{\volt} eingestellt, kann aber extern verändert werden.
 

\subsection{Simulation}
Um ein besseres Verständnis für den Multiplizierer zu gewinnen, wird dieser in KiCad mit ngspice simuliert. Als erster Test werden als Input-Quellen zwei Gleichspannungen verwendet. Diese werden mit den Pins $X_1$ und $Y_1$ verbunden. Für die spätere Funktion des \gls{pd} werden die Pins $X_2$ und $Y_2$ an Ground angeschlossen. In KiCad können die Simulationsdaten in eine .raw-Datei exportiert werden, die es möglich macht, die Ergebnisse in Python zu plotten.\par



\begin{figure}[H]
    \centering
    \begin{tikzpicture}
        \begin{axis}[
            width=0.95\linewidth,
            height=0.55\linewidth,
            xlabel={Zeit / \si{\milli\second}},
            ylabel={Spannung / \si{\volt}},
            xmin=0, xmax=8,
            xtick={0,1,2,3,4,5,6,7,8},
            grid=both,
            legend cell align=left,
            legend pos=north east,
            title style={yshift=-1.5ex},
            tick label style={font=\footnotesize},
            title={Demonstration mit DC-Spannungen},
        ]
            \addplot[color={rgb,255:red,10;green,85;blue,140}, solid, thick, line width=2]
                table[x=time_ms, y=in1_dc_V, col sep=comma]{Bilder/analog_multiplier_dc_down.csv};
            \addlegendentry{Input $X_1$}
            
            \addplot[color={rgb,255:red,195;green,5;blue,35}, solid, thick, line width=2]
                table[x=time_ms, y=in2_dc_V, col sep=comma]{Bilder/analog_multiplier_dc_down.csv};
            \addlegendentry{Input $Y_1$}
            
            \addplot[color={rgb,255:red,250;green,190;blue,0}, solid, thick, line width=2]
                table[x=time_ms, y=out_dc_V, col sep=comma]{Bilder/analog_multiplier_dc_down.csv};
            \addlegendentry{Output}
        \end{axis}
    \end{tikzpicture}
    \caption{Demonstration des analogen Multipliziers mit DC-Spannungen}
    \label{fig:analog_multiplier_dc}
\end{figure}





%\begin{figure} [H]
%    \centering
%    \includegraphics[width=0.8\linewidth]{../Bilder/dc_multi.png}
%    \caption{Multiplikation von DC-Spannungen}
%    \label{fig:mult_dc}
%\end{figure}





Die Grafik \ref{fig:mult_dc} zeigt, dass die oben beschriebene Gleichung \ref{eq:datasheet_multi2} mit der Simulation übereinstimmt. Es können zudem nicht nur positive, sondern auch negative Spannungen korrekt multipliziert werden.



\section{Multiplizierer als Phasendetektor}
Nach der Analyse des analogen Multiplizierers kann nun der erste Baustein des \gls{pll} untersucht werden. Der \gls{pd} baut auf einem Multiplizierer auf, der die Phasendifferenz zwischen zwei Signalen detektieren soll.\par

\begin{figure}[H]  
  \centering
  \resizebox{0.6\textwidth}{!}{%\documentclass[tikz,border=10pt]{standalone}
%\usepackage{tikz}
%\usetikzlibrary{arrows.meta, positioning}

%\begin{document}

\begin{tikzpicture}
    [  block/.style={draw, fill=white, rectangle, minimum            height=2em, minimum width=3em},
    block2/.style = {draw, fill=white, rectangle, minimum            height=4em, minimum width=6em},
    sum/.style={draw, fill=white, circle, minimum size=2.5em, inner sep=0pt},
       lcircle/.style={draw, fill=white, circle, minimum size= 0.5em, inner sep=0pt},
       connector/.style={-Latex, thick},
       node distance=2cm and 2cm]

\node[lcircle] (input1) {};
\node[lcircle, below=2cm of input1] (input2) {};


\coordinate[below=1cm of input1] (mid);
\node[block2, right=2.5cm of mid] (prod) {$\prod$};
\coordinate[left=0cm of prod] (stutze);

\coordinate[above=0.3cm of stutze] (mult_int1);
\coordinate[below=0.3cm of stutze] (mult_int2);

\coordinate[left=1cm of mult_int1] (hilfe1);
\coordinate[left=1cm of mult_int2] (hilfe2);

\node[lcircle, right=of prod] (output) {};

\draw (input1) -| (hilfe1) node[midway,above, xshift=-3mm] {$V_x= X \cdot sin(\omega t)$};
\draw[connector] (hilfe1) -- (mult_int1);
\draw (input2) -| (hilfe2) node[midway, below, yshift=-0mm] {$V_y= Y \cdot sin(\omega t + \phi)$};
\draw[connector] (hilfe2) -- (mult_int2);

\draw[connector] (prod) -- (output) node[near end, above] {$V_o$};
    
\end{tikzpicture}

%\end{document}}
  \caption{Reaktion des Multiplizierers auf phasenverschobene Eingangssignale}
  \label{fig:bsb_multi}
\end{figure}

In Abbildung~\ref{fig:bsb_multi} ist zu erkennen, wie zwei um den Phasenwinkel $\phi$ versetzte Signale auf die Eingänge des Multiplizierers gelegt werden. Dadurch lässt sich der Ausgang des Multiplizierers $V_o$ durch folgende Gleichung beschreiben


\begin{equation*}
V_o = \frac{X Y}{2 V_r} \cdot [cos(\phi)-cos(2\omega t + \phi)]
\end{equation*}

wobei
\begin{itemize}
  \item $X$ und $Y$ die Amplituden der Eingangssignale sind,
  \item $V_r$ der Referenzwert des Multiplizierers ist (laut Datenblatt: $V_r=\SI{10}{\volt}$),
  \item $\phi$ die Phasendifferenz zwischen den beiden Eingangssignalen beschreibt.
\end{itemize}


\textbf{Hinweis:} Im ASLK Manual steht hier $V_o = \frac{X Y}{2 V_r} \cdot [cos(\phi)-cos(\omega t + \phi)]$ was nicht korrekt ist.\par
\medskip

Die  Multiplikation zweier sinusförmiger Signale ergibt demnach ein Signal mit zwei Frequenzkomponenten. Eine Frequenz ist hierbei eine Gleichspannungskomponente $cos(\phi)$, die sich proportional zur Phasendifferenz verhält. Zusätzlich gibt es noch eine hochfrequente Mischkomponente, die mit der doppelten Frequenz des Eingangssignals schwingt. Wenn der Multiplizierer nicht komplett im linearen Bereich opperiert werden zudem noch weitere Hochfrequenzkomponenten als vielfaches der Ausgangsfrequenz generiert. \cite{chengCom} \par
\medskip
Der zweite Block innerhalb des \gls{pll} ist der Loopfilter. Dieser hat die Aufgabe, die hochfrequenten Anteile der Multiplikation zu unterdrücken. So kann für den Loopfilter beispielsweise ein einfacher RC-Tiefpass verwendet werden. Nach der idealen Tiefpass-Filterung des Ausgangssignals reduziert sich der Ausdruck auf\par

\begin{equation}
    V_o = \frac{X Y}{2 V_r} \cdot cos(\phi)
\label{eq:Multi_vereinfacht}
\end{equation}


Diese Gleichung zeigt die direkte Abhängigkeit zwischen Ausgangsspannung des Multiplizierers und der Phasendifferenz der beiden Eingangssignale. Durch diese Verschaltung wird aus dem Multiplizierer ein \gls{pd}, der bei einer Phasendifferenz von $\SI{90}{\degree}$ eine Durchschnittsspannung von $\SI{0}{\volt}$ ausgibt.\cite{Lab_Kit_PRO} %Diese Abhänigkeit kann im Folgenden dazu verwendet werden die Ausgangsspannung als Steuerspannung für den \gls{vcf} zu verwenden.

Die Abbildung~\ref{fig:av_volt_phase} veranschaulicht die Phasencharakteristik des Multiplizierers.\par
\medskip



\begin{figure}[H]
    \centering
    \begin{tikzpicture}
        \begin{axis}[
            width=0.9\linewidth,
            height=0.5\linewidth,
            xlabel={$\phi$ / \si{\radian}},
            ylabel={$V_\text{av}$ / \si{\volt}},
            grid=both,
            legend cell align=left,
            legend pos=south west,  % besser bei negativen Werten
            tick label style={font=\footnotesize},
            xtick={0, 0.785, 1.57, 2.356, 3.14},
            xticklabels={$0$, $\frac{\pi}{4}$, $\frac{\pi}{2}$, $\frac{3\pi}{4}$, $\pi$},
            %ymin=-10, ymax=10,     % ← Jetzt von -10 bis +10!
        ]
            \addplot[color={rgb,255:red,10;green,85;blue,140}, solid, thick, line width=2]
                table[x=phi_rad, y=V_av, col sep=comma]{Bilder/multiplier_phase_char.csv};
            \addlegendentry{Phasencharakteristik}
        \end{axis}
    \end{tikzpicture}
    \caption{Phasencharakteristik des Multiplizierers}
    \label{fig:av_volt_phase}
\end{figure}


%\begin{figure}[H]
%    \centering
%    \includegraphics[width=0.8\linewidth]{../Bilder/phasencharakteristik_multi.png}
%    \caption{Duchschnittliche Ausgangsspannung des Multiplizierers gegen den Phasenwinkel \cite{YT_stf_lecture23}}
%    \label{fig:av_volt_phase}
%\end{figure}

\textbf{kann es sein dass die y-Achse von den Werten her falsch beschriftet ist? Das Maximum dieser Kennlinie sollte doch von $\frac{max_x \cdot max_y}{V_r}$ abhängen.}

Für den alleinstehenden Multiplizierer führt eine Phasendrehung von \SI{0}{\degree} zu einer positiven Spannung am Ausgang. Bei einer Phasendrehung von \SI{180}{\degree} sollte die Ausgangsspannung negativ sein. \cite{YT_stf_lecture23} \par
\medskip

Damit bleibt das Problem, dass der Detektor nur eine Phasendifferenz von genau \SI{90}{\degree} erkennen kann, da der Gleichspannungsanteil des Ausgangssignals an dieser Stelle \SI{0}{\volt} beträgt. Einer andere konstante Phasenverschiebung kann zwar detektiert werden, jedoch nicht als Arbeitspunkt für die Regelung fungieren, sodass sich das System auf diesen Punkt einpendelt/lockt. Die Verschiebung zwischen dem Referenzsignal und dem internen Signal muss also zwangsläufig \SI{90}{\degree} betragen. Dafür wird nun innerhalb des Biquads nach einem solchen Signal gesucht.\par
\medskip

Bei der Auswahl des internen Signals soll sich die Phase um die Mittenfrequenz $\omega=\omega_0$ um \SI{90}{\degree} gegenüber dem Eingangssignal unterscheiden. In Frage kommen daher sowohl eine Phasenverschiebung von \SI{90}{\degree} als auch von \SI{-90}{\degree}, wobei \SI{-90}{\degree} auch als \SI{270}{\degree} interpretiert werden kann. Das Eingangssignal dient dabei als Bezugssignal und definiert die Referenzphase von \SI{0}{\degree}.\par
\medskip


\begin{figure}[H]
    \centering
    \begin{tikzpicture}
        \begin{axis}[
            width=0.9\linewidth,
            height=0.55\linewidth,
            xlabel={Frequenz / \si{\hertz}},
            ylabel={Phase / \si{\radian}},
            xmode=log,
            grid=both,
            legend cell align=left,
            legend pos=north east,
            tick label style={font=\footnotesize},
            ymin=-3.5, ymax=3.5,  
            ytick={-3.14, -1.57, 0, 1.57, 3.14},
            yticklabels={$-\pi$, $-\frac{\pi}{2}$, $0$, $\frac{\pi}{2}$, $\pi$},
        ]
            \addplot[color={rgb,255:red,10;green,85;blue,140}, solid, thick, line width=2]
                table[x=freq, y=lp, col sep=comma]{Bilder/biquad_phase.csv};
            \addlegendentry{Tiefpass}
            \addplot[color={rgb,255:red,195;green,5;blue,35}, solid, thick, line width=2]
                table[x=freq, y=hp, col sep=comma]{Bilder/biquad_phase.csv};
            \addlegendentry{Hochpass}
            \addplot[color={rgb,255:red,0;green,145;blue,90}, solid, thick, line width=2]
                table[x=freq, y=bp, col sep=comma]{Bilder/biquad_phase.csv};
            \addlegendentry{Bandpass}
            \addplot[color={rgb,255:red,250;green,190;blue,0}, dashed, thick, line width=2]
                table[x=freq, y=bs, col sep=comma]{Bilder/biquad_phase.csv};
            \addlegendentry{Bandstop}
        \end{axis}
    \end{tikzpicture}
    \caption{Phasengang der vier Ausgänge des Biquads}
    \label{fig:phaseshift}
\end{figure}


%\begin{figure} [H]
%    \centering
%    \includegraphics[width=0.8\linewidth]{../Bilder/phasengang_biquad_sem6.png}
%    \caption{Phasengänge der vier Ausgänge des Biquads}
%    \label{fig:phaseshift}
%\end{figure}


Der Biquad besteht aus vier Filtertypen deren Phasengänge sich deutlich von einander Unterscheiden. Bei genauerer Betrachtung der Phasenverläufe in Abbildung~\ref{fig:phaseshift} der Filtertypen fällt auf, dass Hoch- und Tiefpass um $\omega_0$ eine Phasenverschiebung von \SI{90}{\degree} bzw.~\SI{-90}{\degree} gegenüber dem Eingangssignal aufweisen. Der Bandpassfilter hat in dieser Umgebung eine Phasenverschiebung von \SI{180}{\degree} und die Bandsperre hat einen Phasensprung. Damit erfüllen sowohl Tiefpass- als auch Hochpassausgang die Bedingung einer konstanten \SI{90}{\degree}-Phasendifferenz, sodass der \gls{pd} bei richtiger Abstimmung in beiden Fällen einen Mittelwert von \SI{0}{\volt} am Ausgang liefern sollte.
\medskip

\begin{figure}[H]
  \centering
  \resizebox{0.8\textwidth}{!}{%\documentclass[varwidth=true, border=10pt, crop=true]{standalone}
%\usepackage{tikz}
%\usepackage{circuitikz}
%\usetikzlibrary{positioning,arrows.meta}

%\begin{document}


 \begin{circuitikz}[european, block2/.style = {draw, fill=white, rectangle, minimum height=3em, minimum width=4.5em}, lcircle/.style={draw, fill=white, circle, minimum size= 0.25em, inner sep=0pt}, node distance=2cm and 2cm]


% === Multiplikator ===
\node[lcircle] (input1) {};
\node[lcircle, below=2cm of input1] (input2) {};


\coordinate[below=1cm of input1] (mid);
\node[block2] (prod) [right=2cm of mid] {$\prod$};
\coordinate[left=0cm of prod] (stutze);

\coordinate[above=0.3cm of stutze] (mult_int1);
\coordinate[below=0.3cm of stutze] (mult_int2);

\coordinate[left=1cm of mult_int1] (hilfe1);
\coordinate[left=1cm of mult_int2] (hilfe2);

\draw (input1) -| (hilfe1) node[near start,above, xshift=-3mm] {$V_{HP}$};
\draw (hilfe1) -- (mult_int1);
\draw (input2) -| (hilfe2) node[near start, below, xshift=-3mm] {$V_{in}$};
\draw (hilfe2) -- (mult_int2);




% === OP === 

\coordinate[right=0cm of prod] (start);
\coordinate[right=2.5cm of start] (endr);
\draw (start) to[R=R, -*] (endr);
\node[op amp, right=0cm of endr, anchor=-] (opamp) {};

\draw (opamp.+) -- ++ (0,-0.86) node[ground] {};

\coordinate (vout) at (opamp.out);
\coordinate[above=1.5cm of endr] (helpC);
\coordinate[above=2cm of vout] (startr2);
\draw (vout) -- (startr2) to[C, l=$C$] (helpC) -- (endr);

\coordinate[right=2.5cm of startr2] (endr2);
\draw (startr2) to[R=R,*-*] (endr2);
\coordinate[right=0.75cm of endr2] (Vc);
\draw (endr2) to[short,-o]  (Vc) node[above] {$V_c$};


\coordinate[below=2cm of endr2] (endr3);
\draw (endr2) to [R=R] (endr3);
\coordinate[below=1.35cm of endr3] (gnd2);
\draw (endr3) to[V=$V_H$] (gnd2);
\draw (gnd2) node[ground] {};


\end{circuitikz}


%\end{document}
}
  \caption{Teilschaltung: Phasendetektor (PD)}
  \label{fig:sb_phasedetector}
\end{figure}

Hinter dem Multiplizierer befindet sich, wie in Abbildung~\ref{fig:sb_phasedetector} zusehen, ein Integrator. Dieser hat zum einen die Aufgabe, die hochfrequenten Anteile des Multiplizierters durch seine Tiefpasscharakteristik herauszufiltern. Zum anderen integriert dieser das eingehende DC-Signal, sodass die Ausgangsspannung bei langanhaltender, großer Phasendifferenz im Bezug zu \SI{\pm 90}{\degree} immer größer wird.\par
\medskip

Nach dem \gls{opv} befindet sich im Schaltplan des ALSK-PRO Manuals noch eine Teilschaltung bestehend aus Spannungsquelle $V_H$ und zwei Widerständen. Leider wird dabei weder der Zweck noch die Höhe der Hilfsspannung genauer erläutert, woduch es schwerer fällt diesen Schaltungsteil zu verstehen.\par
\medskip
Zu den anfänglichen Überlegungen bezüglich der Funktion von $V_H$ gehörte die Annahme, dass die Ausgangsspannung des Integrators zur Verwendung als Steuerspannung für den \gls{vco} auf ein geeignetes Potential angehoben werden muss. Diese Vermutung begründet sich durch die Verschaltung der internen Integratoren da der Rückführungspfad dieser durch einen Multiplizierer erweitert wird. Dadurch wird aus der ursprünglichen Beziehung

\begin{equation*}
    V_{cap} = V_{out}
\end{equation*}
  
durch den Multiplizierer die Beziehung
  
\begin{equation*}
    V_{cap} = \frac{V_{out} \cdot V_{c}}{V_r}
\end{equation*}

Soll nun also der ursprüngliche Zustand wieder hergestellt werden, müsste die Steuerspannung $V_c$ \textbf{im Durchschnitt?} auf das bekannte Referenzpotential $V_r=\SI{10}{\volt}$ angehoben werden, um den Bruch zu kompensieren.\par
\medskip
Aus dieser Überlegung ergeben sich mehrere Probleme. Zum einen liegt die Ausgangsspannung des Integrators bei etwa \SI{2.5}{\volt}. Um diesen Wert über die gezeigte Schaltung anzuheben, müsste $V_H$ eine sehr hohe Spannung (deutlich über \SI{10}{\volt}) haben, da die Widerstände einen Spannungsteiler bilden. Zum anderen stellt sich die Frage, ob die Phaseninformation noch erhalten bleibt oder ob ein Teil des Stroms zurück in den \gls{pd} fließt und so die Ausgangsspannung beeinflusst. Um die Steuerspannung auf das gewünschte Potential anzuheben, ist diese Verschaltung daher eher ungeeignet.\par
\medskip
Auf die tatsächliche Funktion der Steuerspannung im \gls{vcf} wird im Kapitel~\ref{sec:VCF} genauer eingegangen. Der tatsächliche Nutzen der Hilfsspannungsquelle wird warscheinlich die Stromverstärkung des Kontrollsignals sein. Dabei muss $V_H$  selbst keinen hohen Wert besitzen, was besser zu der Charakteristik einer Hilfsspannungsquelle passt. Auf diese Weise können die internen Multiplizierer das Steuersignal besser verarbeiten.\par
\medskip
So steht im nächsten Schritt eine geeingete DC-Steuerspannung $V_c$ für die internen Multiplizierer im \gls{vcf} zur Verfügung, die das Signal durch die Stromverstärkung besser erfassen können.\par
\medskip

%%

\subsection{Simulation von Eingangssignalen mit Phasenverschiebung}
Im folgenden Abschnitt werden die zuvor beschriebenen Zusammenhänge durch Simulationen überprüft. Hierfür werden dem System Wechselspannungen mit unterschiedlichen Phasenlagen zugeführt.\par
\medskip

Am Eingang $Y_1$ liegt immer ein Sinussignal an. Am Eingang $X_1$ wird das gleiche Signal mit einer veränderten Phase eingespeißt. Im ersten Fall bleibt die Phase unverändert ($\phi = \SI{0}{\degree}$), im zweiten Fall wird sie um \SI{90}{\degree} und im dritten Fall um \SI{180}{\degree} verschoben. Da der Arbeitsbereich der Schaltung bei einem Phasenversatz von \SI{90}{\degree} liegt und das Sinussignal periodisch ist, stellen Phasenverschiebungen von \SI{0}{\degree} und \SI{180}{\degree} die maximal möglichen Abweichungen dar. Der Idealwert wird bei $\phi = \SI{90}{\degree}$ bzw. \SI{270}{\degree} der erreicht. Die real auftretenden Werte sollten daher zwischen oder auf diesen Extrempunkten liegen.\par
\medskip

Im linken Teil der Abbildung~\ref{fig:ac_multi_pd} sind die drei untersuchten Eingangssignale als Zeitverläufe dargestellt. Der rechte Teil zeigt die zugehörigen Ausgangssignale des Multiplizierers.\par
\medskip

\begin{figure}[H]
    \centering
    \begin{tikzpicture}
        \begin{groupplot}[
            group style={
                group size=2 by 1,
                horizontal sep=1.5cm, % Etwas mehr Platz für das ylabel des rechten Plots
            },
            width=0.495\linewidth,
            height=0.6\linewidth,
            grid=both,
            legend cell align=left,
            legend pos=south west,
            xlabel={Zeit / \si{\milli\second}},
            xmin=0, xmax=2,
            % --- Korrekturen ---
            ylabel style={xshift=-0.5cm}, % Schiebt das y-Label nach links
            title style={yshift=-1.5ex},  % Schiebt den Titel näher an den Plot
            % -------------------
            yticklabel pos=left,
            ytick style={black},
            tick label style={font=\footnotesize},
        ]
        \nextgroupplot[
            ylabel={Spannung / \si{\volt}},
            title={Eingangssignale},
        ]
            \addplot[color={rgb,255:red,10;green,85;blue,140}, solid, thick, line width=2]
                table[x=time, y=in0, col sep=comma]{Bilder/multiplier_inputs.csv};
            \addlegendentry{$\phi=0^\circ$}
            \addplot[color={rgb,255:red,195;green,5;blue,35}, solid, thick, line width=2]
                table[x=time, y=in90, col sep=comma]{Bilder/multiplier_inputs.csv};
            \addlegendentry{$\phi=90^\circ$}
            \addplot[color={rgb,255:red,250;green,190;blue,0}, solid, thick, line width=2]
                table[x=time, y=in180, col sep=comma]{Bilder/multiplier_inputs.csv};
            \addlegendentry{$\phi=180^\circ$}

        \nextgroupplot[
            % Falls der zweite Plot auch ein ylabel braucht, hier einkommentieren:
            % ylabel={Spannung / \si{\volt}}, 
            title={Ausgangssignale},
        ]
            \addplot[color={rgb,255:red,10;green,85;blue,140}, solid, thick, line width=2]
                table[x=time, y=out0, col sep=comma]{Bilder/multiplier_outputs.csv};
            \addlegendentry{$\phi=0^\circ$}
            \addplot[color={rgb,255:red,195;green,5;blue,35}, solid, thick, line width=2]
                table[x=time, y=out90, col sep=comma]{Bilder/multiplier_outputs.csv};
            \addlegendentry{$\phi=90^\circ$}
            \addplot[color={rgb,255:red,250;green,190;blue,0}, solid, thick, line width=2]
                table[x=time, y=out180, col sep=comma]{Bilder/multiplier_outputs.csv};
            \addlegendentry{$\phi=180^\circ$}
        \end{groupplot}
    \end{tikzpicture}
    \caption{Signalverhalten bei unterschiedlichen Phasenlagen zwischen den Eingangssignalen $X_1$ und $Y_1$}
    \label{fig:ac_multi_pd}
\end{figure}


%\begin{figure}[H]
%  \centering
%    \includegraphics[width=1\linewidth]{../Bilder/ac_multi_pd.png}
%    \caption{Signalverhalten bei unterschiedlichen Phasenlagen zwischen den Eingangssignalen $X_1$ und $Y_1$}
%    \label{fig:ac_multi_pd}
%\end{figure}


Wie erwartet weist das Signal mit einer Phasenverschiebung von \SI{90}{\degree} nach der Multiplikation einen Mittelwert von \SI{0}{\volt} auf. Das unverschobene Signal zeigt einen Offset von etwa \SI{0.2}{\volt}, während das um \SI{180}{\degree} verschobene Signal einen Offset von \SI{-0.2}{\volt} besitzt. Die in Abbildung \ref{fig:av_volt_phase} gezeigte Kennlinie kann somit von der allgemeinen Form her, \SI{0}{\degree} Phasenverzug ergibt positive durchschnittliche Ausgangsspannung, \SI{90}{\degree} hat keinen Offset und \SI{180}{\degree} ergibt einen negativen Offset, bestätigt werden. Bei allen drei Signalen enthält das Ausgangssignal einen hochfrequenten Anteil mit der doppelten Frequenz des Eingangssignals.\par
\medskip

Laut Datenblatt des MPY634 ergibt sich eine Phasendetektorschaltung, wenn am Ausgang des Multiplizierers ein einfacher RC-Tiefpass nachgeschaltet wird. In anderen Schaltungsvarianten wird am Multipliziererausgang ein Tiefpass mit anschließendem \gls{opv} in Komparatorschaltung verwendet. Bei alleiniger Nutzung des RC-Tiefpass wird allein die Restwelligkeit unterdrückt, sodass der Mittelwert des Ausgangssignals die Ausgangsspannung bildet.\textbf{Überarbeiten}\par
\medskip

Der Schaltungsaufbau im ALSK-Manual sieht hingegen vor, dass am Ausgang des Multiplizierers ein Integrator nachgeschaltet wird, der den Schleifenfiter representiert. Dieser verhält sich ebenfalls wie ein Tiefpass. So entstehen aus den in Abbildung~\ref{fig:ac_multi_pd} sichtbaren Signalen nach der Integration die in Abbildung~\ref{fig:pd_op_out} dargestellten Signalverläufe.\par
\medskip


\begin{figure}[h!]
    \centering
    \begin{tikzpicture}
        \begin{groupplot}[
            group style={
                group size=2 by 1,
                horizontal sep=1.2cm, % Etwas schmaler, da rechts kein ylabel steht
            },
            width=0.495\linewidth,
            height=0.6\linewidth,
            grid=both,
            legend cell align=left,
            legend pos=south west,
            xlabel={Zeit / \si{\milli\second}},
            xmin=0, xmax=10,
            ymin=0, ymax=4.75, 
            ylabel style={xshift=-0.5cm}, 
            title style={yshift=-1.5ex},
            yticklabel pos=left,
            ytick style={black},
            tick label style={font=\footnotesize},
        ]
        \nextgroupplot[
            ylabel={Spannung / \si{\volt}},
            title={$R = \SI{1}{\kilo\ohm}$},
        ]
            \addplot[color={rgb,255:red,10;green,85;blue,140}, solid, thick, line width=2]
                table[x=time, y=detec0, col sep=comma]{Bilder/pd_detector_outputs.csv};
            \addlegendentry{$\phi=0^\circ$}
            \addplot[color={rgb,255:red,195;green,5;blue,35}, solid, thick, line width=2]
                table[x=time, y=detec90, col sep=comma]{Bilder/pd_detector_outputs.csv};
            \addlegendentry{$\phi=90^\circ$}
            \addplot[color={rgb,255:red,250;green,190;blue,0}, solid, thick, line width=2]
                table[x=time, y=detec180, col sep=comma]{Bilder/pd_detector_outputs.csv};
            \addlegendentry{$\phi=180^\circ$}
        \nextgroupplot[
            title={$R = \SI{5}{\kilo\ohm}$},
            % ylabel weggelassen
        ]
            \addplot[color={rgb,255:red,10;green,85;blue,140}, solid, thick, line width=2]
                table[x=time, y=op0, col sep=comma]{Bilder/pd_op_outputs.csv};
            \addlegendentry{$\phi=0^\circ$}
            \addplot[color={rgb,255:red,195;green,5;blue,35}, solid, thick, line width=2]
                table[x=time, y=op90, col sep=comma]{Bilder/pd_op_outputs.csv};
            \addlegendentry{$\phi=90^\circ$}
            \addplot[color={rgb,255:red,250;green,190;blue,0}, solid, thick, line width=2]
                table[x=time, y=op180, col sep=comma]{Bilder/pd_op_outputs.csv};
            \addlegendentry{$\phi=180^\circ$}
            
        \end{groupplot}
    \end{tikzpicture}
    \caption{Signalverhalten bei unterschiedlicher Verschaltung des Integrators}
    \label{fig:pd_op_out}
\end{figure}

%\begin{figure} [h!]
%    \centering
%    \includegraphics[width=1\linewidth]{../Bilder/pd_op_out.png}
%    \caption{Signalverhalten bei unterschiedlicher Verschaltung des Integrators}
%    \label{fig:pd_op_out}
%\end{figure}

Die Abbildung~\ref{fig:pd_op_out} zeigt den zeitlichen Verlauf des Integratorausgangs für die verschiedenen Phasenverschiebungen. Zu beachten ist hierbei, dass die Phasenlage der Eingangssignale unter realen Bedingungen nicht über längere Zeit auf den Maximalwerten $\phi = \SI{0}{\degree}$ bzw.~$\phi = \SI{180}{\degree}$  bleibt, sondern sich dynamisch verändert. Für $\phi = \SI{90}{\degree}$ wird die Amplitude der hochfrequenten Komponente nach der Integration deutlich gedämpft, dennoch bleibt eine Restschwingung sichtbar. Die Mittelwertspannung bleibt in diesem Fall über die Zeit gleich, da die Summation der positiven und negativen Halbwellen des Eingangssignals zu 0 verläuft. Auffällig ist, dass dem Signal eine Gleichspannungskomponente von etwa \SI{2.51}{\volt} hinzugefügt wurde. Dies ist auf das Integrationsverhalten und die Verschaltung des Integrators zurückzuführen.\par
\medskip
Diese Gleichspannungskomponente kann auf verschiedene Ursachen zurückgeführt werden. Beispielsweise enthält das Simulationsmodell des TL082 Startbedingungen (Initial Bias), die an internen Transistorknoten VC und VE ein Potential von \SI{2.2}{\volt} als Startwert definiert. Dadurch lässt sich bereits ein großer Teil des Offsets erklären. Die restlichen \SI{0.3}{\volt} könnten durch eine Standard-Eingangsoffsetspannung kommen. Diese wird sofort in der Integration berücksichtigt und führt zu einer Gleichspannung am Ausgang, obwohl rein mathematisch kein Offset vorhanden sein sollte. Die Simulation mit einem idealen OPV sollte diese zusätzliche Verstärkung also nicht zeigen.
\medskip
Bei den Extremwerten der Phasenverschiebung $\phi = \SI{0}{\degree}$ und  $\phi = \SI{180}{\degree}$ zeigt sich ebenfalls eine Erhöhung des DC-Anteils und eine gedämpfte Amplitude der AC-Komponente. Da das Sinussignal für beide Fälle nicht mehr um \SI{0}{\volt} zentriert ist, summieren sich die Schwingungen beim integrieren immer weiter auf. Für  $\phi = \SI{0}{\degree}$ steigt das Ausgangssignal linear mit einer Steigung von \SI{0.2}{\volt\per\milli\second} an, für $\phi = \SI{180}{\degree}$ fällt die Spannung mit gleicher negativer Steigung ab. Zusammenfassend lässt sich festhalten, dass der Integrator durch seine Tiefpass-Charakteristik ebenfalls die Welligkeit unterdrückt. Durch die Integration wird allerdings zudem der Phasenfehler aufsummiert, was, wie in Abbildung \ref{fig:pd_op_out} zusehen, dazu führen kann, dass das Ausgangssignal bei einem konstanten Phasenfehler driftet. \par
\medskip
Ausgehend davon, dass die AC-Komponente noch deutlich sichtbar ist, kann die Amplitude durch Reduzierung der Filter-Mittenfrequenz über den Vorwiderstand weiter verringert werden. Dies führt, wie im rechten Bild zusehen, zu einer weiteren Reduktion der Restwelligkeit. Bei genauer Betrachtung ist jedoch zu erkennen, dass immernoch eine gewisse Restwelligkeit vorhanden ist. Eine kurze Berechung in Python bestätigt, dass die Restwelligkeit des Signals um einen Faktor von \num{4.97} reduziert wird. Außerdem fällt auf, dass die Steigung bei einer niedrigen Mittenfrequenz des Filters deutlich geringer ausfällt als bei einer hohen Mittenfrequenz. Diese liegt mit \SI{0.04}{\volt\per\milli\second} genau um den Faktor~\num{5} geringer als die Vergleichs-Ausgangsspannung. Damit beeinflusst die Mittenfrequenz des Integrators maßgeblich die Genauigkeit (Sensitivität) des \gls{pd}.\par
\medskip
Zusätzlich lässt sich beobachten, dass sich bei sinkender Mittenfrequenz des Tiefpassfilters auch die zuvor erwähnte Gleichspannungs-Komponente ändert. In den vorliegenden Messungen steigt dieser Wert leicht auf \SI{2,565}{\volt}. \textbf{Dieser Effekt ist so zu beschreiben. Der Spannungsabfall über den Vorwiderstand ist mit $V=I_{Bias} \cdot R$ zu beschreiben. Dabei ist der Eingangsbiasstrom vom OPV abhängig und immer gleich groß. Je größer also der Vorwiderstand R wird, desto größer ist der Spannungsabfall über diesen, was ebenfalls zu einem höheren Offset am ausgang führt. Zudem ist die Gesammtverstärkung des OPVs $\frac{1}{RC}$, wenn $R$ nun größer wird, wird der Offset größer??? }

\medskip
%https://www.ti.com/content/dam/videos/external-videos/zh-tw/1/3816841626001/4082104055001.mp4/subassets/opamps-offset-voltage-input-bias-specifications-presentation-quiz.pdf


\subsection{Simulation von Eingangssignalen mit unterschiedlichen Frequenzen}
Ziel dieser Simulation ist, das Verhalten der Steuerspannung $V_c$ bei unterschiedliche Phasendifferenzen am Eingang des \gls{pd} zu untersuchen. Dafür werden zwei Signale ähnlicher Frequenz auf das System gegeben. Das Referenzsignal schwingt mit \SI{1000}{\hertz}, dazu kommt ein Eingangssignal mit %\SI{900}{\hertz} bzw.
\SI{1100}{\hertz}.\par
\medskip


\begin{figure}[H]
    \centering
    \begin{tikzpicture}
        \begin{groupplot}[
            group style={
                group size=1 by 2,
                vertical sep=1.2cm, % Etwas mehr Platz für die Zahlen des oberen Plots
            },
            width=0.95\linewidth,
            height=0.22\textheight,
            grid=both,
            legend cell align=left,
            legend pos=south west,
            xlabel={Zeit / \si{\milli\second}},
            xmin=0, xmax=10,
            ylabel style={xshift=-0.5cm}, 
            title style={yshift=-1.5ex},
            tick label style={font=\footnotesize},
            every axis y label/.style={at={(ticklabel cs:0.5)}, rotate=90, anchor=near ticklabel},
            yticklabel style={text width=2.5em, align=right}, % Reserviert festen Platz
        ]
        
        % Erster Plot: Eingangssignale (Zahlen bleiben, Label "Zeit" verschwindet)
        \nextgroupplot[
            ylabel={Spannung / \si{\volt}},
            title={Eingangssignale des Multiplizierers  in $X_1$},
            xlabel={}, % Nur das Label (der Text) wird hier gelöscht
        ]
            \addplot[color={rgb,255:red,10;green,85;blue,140}, solid, thick, line width=2]
                table[x=time, y=in1000, col sep=comma]{Bilder/mult_freq_inputs.csv};
            \addlegendentry{$f=1000\,\si{\hertz}$}
            \addplot[color={rgb,255:red,0;green,145;blue,90}, solid, thick, line width=2]
                table[x=time, y=in1100, col sep=comma]{Bilder/mult_freq_inputs.csv};
            \addlegendentry{$f=1100\,\si{\hertz}$}

        % Zweiter Plot: Ausgangssignale
        \nextgroupplot[
            ylabel={Spannung / \si{\volt}},
            title={Ausgangssignale des Multiplizierers},
        ]
            \addplot[color={rgb,255:red,10;green,85;blue,140}, solid, thick, line width=2]
                table[x=time, y=out1000, col sep=comma]{Bilder/mult_freq_outputs.csv};
            \addlegendentry{$f=1000\,\si{\hertz}$}
            \addplot[color={rgb,255:red,0;green,145;blue,90}, solid, thick, line width=2]
                table[x=time, y=out1100, col sep=comma]{Bilder/mult_freq_outputs.csv};
            \addlegendentry{$f=1100\,\si{\hertz}$}

        \end{groupplot}
    \end{tikzpicture}
    \caption{Signalverhalten bei unterschiedlichen Eingangsfrequenzen}
    \label{fig:pd_out_freq}
\end{figure}


%\begin{figure} [H]
%    \centering
%    \includegraphics[width=1\linewidth]{../Bilder/versch_freq_4.png}
%    \caption{Signalverhalten bei unterschiedlichen Eingangsfrequenzen}
%    \label{fig:pd_out_freq}
%\end{figure}

\medskip
Zu Beginn der Simulation beträgt die Phasendrehung des \SI{1100}{\hertz}-Signals \SI{0}{\degree} relativ zum Referenzsignal. Da das Eingangssignal schneller schwingt, verändert sich das Phasenverhältnis in kurzer Zeit (etwa \SI{2.5}{\milli\second}) auf \SI{+90}{\degree}. Das ist unter anderem auch daran zu erkennen, dass der Gleichspannungsanteil im Ausgangssignal des Multiplizierers zu \SI{0}{\volt} abfällt. Aus Abbildung~\ref{fig:ac_multi_pd} ist bekannt, dass bei einer Gleichspannung von \SI{0}{\volt} die Phasenverschiebung \SI{\pm 90}{\degree} beträgt. Da in diesem Simualtionszenario noch keine Anpassung erfolgt, verschiebt sich der Phasenwinkel zwischen den Signalen weiter bis diese bei \SI{5}{\milli\second} \SI{180}{\degree} zu einander stehen. An diesem Punkt ist der maximale negative Gleichspannungsanteil erreicht. Im weiteren Verlauf bewegt sich das Ausgangssignal wieder auf eine \SI{0}{\degree} Phasendifferenz zu. \par
\medskip
Ähnlich verhält sich ein langsameres \SI{900}{\hertz}-Signal. Einziger großer Unterschied ist, dass das Eingangssignal dem Ausgangsssignal nicht voraus läuft, sondern hinterher. Die Phasenverschiebung ist demnach negativ. Das bewirkt auch, dass zwischen dem Kreislauf von \SI{0}{\degree}  zu \SI{0}{\degree} (Periode des niederfrequenten Signal) eine Periode des höherfrequenten Signal weniger im Vergleich zu zweifach multiplizierten Ref-Signals bzw. zwei Perioden weniger als das \SI{1100}{\hertz}-Signal. Außerdem ist durch die negative Phasenverschiebung die Richtung der Phasenverschiebung umgekehrt.(von \SI{0}{\degree} nach \SI{270}{\degree} nach \SI{180}{\degree} nach \SI{90}{\degree} nach \SI{0}{\degree}) \par
\medskip
Aus Abbildung~\ref{fig:av_volt_phase} geht hervor, dass sich die durchschnittliche Ausgangsspannung über einen Phasenverlauf linear verändert. Bei Betrachtung von \ref{fig:pd_out_freq} fällt allerdings auf, dass dieser Verlauf eher leicht sinusförmig zu sein scheint, anstatt linear zu verlaufen. \par
\medskip


\begin{figure}[h!]
    \centering
    \begin{tikzpicture}
        \begin{axis}[
            width=0.95\linewidth, height=0.55\linewidth,
            grid=both, grid style={dashed, gray!30},
            legend pos=south east,
            xlabel={Zeit / \si{\milli\second}}, ylabel={Spannung / \si{\volt}},
            xmin=0, xmax=10,
            title style={yshift=0.8cm},
            title={Eingangssignale des Multiplizierers  in $X_1$},
            % Zweite Achse oben
            extra x ticks={0, 2.5, 5, 7.5, 10},
            extra x tick labels={$0$, $\pm\frac{\pi}{2}$, $\pm \pi$, $\pm \frac{3 \pi}{2}$, $2\pi$},
            extra x tick style={xticklabel pos=right, grid=none, tick label style={yshift=0.1cm}},
        ]
        \addplot[color={rgb,255:red,195;green,5;blue,35}, solid, line width=1.5pt] 
                table[x=time, y=detec900, col sep=comma]{Bilder/detec_freq_outputs.csv};
        \addplot[color={rgb,255:red,10;green,85;blue,140}, solid, line width=1.5pt] 
                table[x=time, y=detec1000, col sep=comma]{Bilder/detec_freq_outputs.csv};
        \addplot[color={rgb,255:red,0;green,145;blue,90}, solid, line width=1.5pt] 
                table[x=time, y=detec1100, col sep=comma]{Bilder/detec_freq_outputs.csv};
        \legend{900 Hz, 1000 Hz, 1100 Hz}
        \end{axis}
    \end{tikzpicture}
    \caption{Abhängigkeit zwischen der Steuerspannung $V_c$ und der Phasendifferenz}
    \label{fig:op_out_freq}
\end{figure}


%\begin{figure}[H]
%  \centering
%    \includegraphics[width=1\linewidth]{../Bilder/versch_freq_3.png}
%    \caption{Abhängigkeit zwischen der Steuerspannung $V_c$ und der Phasendifferenz}
%    \label{fig:op_out_freq}
%\end{figure}
%\textbf{x-achse noch in gradzahlen? ... ja!}

Da nun bekannt ist, bei welchem Zeitpunkt welche Phasenlage zu erwarten ist, kann die x-Achsenbeschriftung durch diese Phasenlagen ersetzt werden. Nach der Integration der Ausgangssignale aus Abbildung~\ref{fig:pd_out_freq}, dass $V_c$ zwischen der Phasenlage von \SI{90}{\degree} über \SI{180}{\degree} bis \SI{270}{\degree} vom Potential her steigt.
(also um \SI{180}{\degree}, \SI{90}{\degree} und \SI{270}{\degree} sind die Grenzwerte). Wenn sich die Phase des Eingangssignals aber zwischen \SI{270}{\degree} und \SI{90}{\degree} bewegt, sinkt die Spannung $V_c$.

\section{Aufbau und Steuerung des Voltage Controlled Filters}
\label{sec:VCF}

\subsection{Abbinder zu Phasenregelschleifen}
\textbf{weiß noch nicht wo das hinsoll}
Wie zuvor in Abbildung~\ref{fig:bsb_pll} zu sehen besteht der klassische \gls{pll} aus einem \gls{pd}, einem Schleifenfilter und einem \gls{vco}. Das Schaltbild in Abbildung~\ref{fig:sb_phasedetector} zeigt dabei die ersten zwei Teilmodule. Die Phasendifferenz wird durch den analogen Multiplizierer detektiert und der anschließende Integator filtert die hochfrequente Komponente heraus. Normalerweise würde als nächstes der \gls{vco} folgen. In dieser Arbeit wird allerdings kein \gls{vco} an die Kontrollspannung $V_c$ angeschlossen, sondern ein \gls{vcf}. Also wird im Experiment~\num{5} des ASLK-PRO Manuals kein klassischer \gls{pll} aufgebaut, sondern eine selbstabstimmende Filterstruktur, bei der die Mittenfrequenz des Filters dynamisch an die Frequenz des Eingangssignals angepasst wird.\par
\medskip
Zusammengefasst besteht der Unterschied darin, dass nicht die Frequenz des Oszillators, sondern das Filterverhalten geregelt wird. Trotzdem ähneln sich die Rückkopplungslogik und die mathematische Grundstruktur des \gls{pll} sehr. Die Schaltung basiert somit auf \gls{pll}-Prinzipien, regelt jedoch einen \gls{vcf} anstatt eines \gls{vco}.\par
\medskip


\subsection{Voltage Controlled Filter}
Der spannungsgesteuerte Filter ist ein Filter, bei dem sich die Grenzfrequenz oder andere Filterparameter über eine Steuerspannung verändern lassen. Dadurch ist es möglich, besonders schnell und flexibel auf unterschiedliche Eingangssignale zu reagieren. In der Audiotechnik werden solche Filter häufig in Synthesizern verwendet, um den Klangcharakter von Signalen dynamisch zu verändern.\textbf{Wikipedia englisch}\par
\medskip
Der Voltage Controlled Filter basiert auf dem Biquad aus Kapitel~\ref{sec:theorie1}, zu sehen in Abbildung~\ref{fig:Multiple-Feetback-Biquad}. Neben dem im vorherigen Unterkapitel besprochenen \gls{pd} wird die Biquad-Schaltung so verändert, dass sich die Mittenfrequenz, und darüber auch die Grenzfrequenz, über die Steuerspannung $V_c$ verändern lässt. Dafür wird der Schaltplan um die frequenzbestimmenden Integratoren verändert.

\begin{figure}[!h]
  \centering
  \resizebox{0.6\textwidth}{!}{%\documentclass[varwidth=true, border=10pt, crop=true]{standalone}
%\usepackage{tikz}
%\usepackage{circuitikz}
%\usetikzlibrary{positioning,arrows.meta}

%\begin{document}


 \begin{circuitikz}[european, block2/.style = {draw, fill=white, rectangle, minimum height=3em, minimum width=4.5em}, lcircle/.style={draw, fill=white, circle, minimum size= 0.25em, inner sep=0pt}, connector/.style={-Latex, thick}, node distance=2cm and 2cm]




% === OP === 

\coordinate (start) ;

\coordinate[right=2.5cm of start] (endr);
\draw (start) to[R=R, o-*] (endr) node[near start, above] {$V_{in}$};
\node[op amp, right=0cm of endr, anchor=-] (opamp) {};

\draw (opamp.+) -- ++ (0,-0.5) node[ground] {};

\coordinate (vout) at ($(opamp.out) + (2.1,0)$);
\draw (opamp.out) -- (vout);
\coordinate[above=1.5cm of endr] (helpC);
\coordinate[above=2cm of vout, xshift = -2.6cm] (startc);
\coordinate[right=0.5cm of startc] (endmulti) ;
\draw (endmulti) -- (startc) node[above]{$V_{cap}$} to[C, a=$C$] (helpC) -- (endr);

\coordinate[right=0.75cm of vout] (Vout_end);
\draw (vout) to[short,*-o] (Vout_end) node[above] {$V_{out}$};


% === Multiplikator ===
\node[block2] (prod) [right=0cm of endmulti] {$\prod$};
\coordinate[right=4.5em of endmulti, yshift= 0.3cm] (in_vc);
\coordinate[right=4.5em of endmulti, yshift= -0.3cm] (in_vout);



\draw (vout) |- (in_vout);
\coordinate (vc) at ($(Vout_end |- in_vc)$);
\draw (in_vc) to[short,-o] (vc) node[above] {$V_c$};

\end{circuitikz}



%\end{document}}
  \caption{Teilschaltung: Spannungsgesteuerter Integrator (VCI)}
  \label{fig:sb_vci}
\end{figure}

Im Rückkopplungspfad der Integratoren wird jeweils ein Multiplizierer eingefügt, der die Ausgangsspannung des \gls{opv}s mit der Steuerspannung $V_c$ multipliziert. Wie schon in bei der Standard-Integratorschaltung wird auch für diese Schaltung die Übertragungsfunktion hergeleitet. Da der Strom durch den Widerstand vollständig durch den Kondensator in der Rückführungsschleife fließen muss, ergibt sich folgender Zusammenhang

\begin{equation}
    I_R = \frac{V_{in}}{R} = -I_C = -C \cdot \frac{dV_{cap}}{dt}
\label{eq:integrator_current}
\end{equation}

Daraus folgt

\begin{equation}
\frac{V_{in}}{R} = -C \cdot \frac{dV_{cap}}{dt}
\label{eq:integrator_relation}
\end{equation}

Mit $V_{cap} = \frac{V_{out} \cdot V_c}{V_r}$ ergibt sich

\begin{equation}
  \frac{V_{in}}{R} = -C \cdot \frac{d}{dt}\left(\frac{V_{out} \cdot V_c}{V_r}\right)
\end{equation}

Durch Integration erhält man den Zusammenhang im Zeitbereich

\begin{equation}
    V_{out}(t) = -\frac{V_r}{V_c} \cdot \frac{1}{RC} \int V_{in}(t) \, dt
\label{eq:integrator_time}
\end{equation}

Im Laplace-Bereich ergibt sich entsprechend

\begin{equation}
V_{out}(s) = -\frac{V_r}{V_cRCs} \, V_{in}(s)
\label{eq:integrator_laplace}
\end{equation}

Somit zeigt die Schaltung das Verhalten eines invertierenden Integrators mit einem Verstärkungsfaktor von $-\frac{V_r}{V_cRC}$. Wegen des zusätzlichen Faktors $V_r/V_c$ mit der variablen Spannung $V_c$ wird ein Aufbau wie dieser auch Voltage Controlled Integrator (VCI) genannt.


%was passiert bei hoher $V_c$ und was bei geringer? kann man den mulit als R oder spannungsquelle oder so auffassen?

%Anmerkung: Bei einer hohen Steuerspannung VcV​ verringert sich der Verstärkungsfaktor $-\frac{V_r}{V_cRC}$​​, was zu einer geringeren Integratorverstärkung führt. Umgekehrt erhöht sich die Verstärkung bei einer niedrigen Steuerspannung Vc​. Der Multiplizierer kann in diesem Kontext als einstellbarer Widerstand aufgefasst werden, da er die effektive Rückkopplung des Integrators beeinflusst."


%%

\subsection{Mittenfrequenzbestimmung des spannungsgesteuerten Filters}
\label{sec:theorie_w0}
%Die Grenzfrequenz, auch Cutoff-Frequenz (\textbf{$\omega_0$ ist die Mittenfrequenz!}) genannt, ist einer der wichtigsten Parameter zur Bestimmung von Filtern. Sie zeigt an, wo sich im Spektrum der Übergangsbereich befindet, der den Durchlassbereich(Passband) von Sperrbereich(Stopband) trennt. So gibt dieser Parameter Auskunft darüber, welche Frequenzen verstärkt oder gedämpft werden, was die wichtigste Eingeschaft eines Filters ist.\par
Die Mittenfrequenz ist ein zentraler Parameter zur Charakterisierung von Filtern. Für einen Bandpass gibt sie an, wo sich der Durchlassbereich im Frequenzspektrum befindet. Dabei entspricht $\omega_0$ der Resonanzfrequenz des Filters und beschreibt somit, wo der Filter seine maximale Übertragungsamplitude erreicht.\par
\medskip
Aus der im letzten Abschnitt hergeleiteten Übertragungsfunktion im Laplace-Bereich \eqref{eq:integrator_laplace} kann nun über die systemtheoretische Betrachtung des Filters auf die Gesammtübertragungsfunktion geschlossen werden. Aus den Übertragungsfunktionen der einzelnen OpAmps lässt sich das in Abbildung~\ref{fig:sb_sys_vcf} zu sehende Blockschaltbild erschließen.


\begin{figure}[!h]
  \centering
  \resizebox{1\textwidth}{!}{\begin{tikzpicture}[
  block/.style={draw, fill=gray!20, rectangle, minimum height=2em, minimum width=3em},
  wcircle/.style={draw, fill=white, circle, minimum size=2.5em, inner sep=0pt},
  bcircle/.style={draw, fill=black, circle, minimum size= 0.5em, inner sep=0pt},
  lcircle/.style={draw, fill=white, circle, minimum size= 0.5em, inner sep=0pt},
  sum/.style={draw, circle, minimum size=1em, inner sep=0pt},
  connector/.style={-Latex, thick},
  node distance=2cm and 2cm
  ]

  % Nodes
  \node[lcircle] (input) at (0,0) {};
  \node[wcircle, right=of input] (gain1) {$-H_0$};
  \node[sum, right=1cm of gain1] (sum1) {$\sum$};
  \node[wcircle, right= 2.25cm of sum1] (gain2) {$-1$};
  \node[sum, right= 1cm of gain2] (sum2) {$\sum$};
  \node[wcircle, right=2.5cm of sum2] (gain_vr1) {$V_r$};
  \node[sum] (prod1) at (14.75,0) {$\prod$};

  \node[block, right=3cm of gain_vr1] (gain3) {$\frac{1}{s}$};
  \node[wcircle, right=1.5cm of gain3] (gain4) {$-\frac{1}{RC}$};
  \node[bcircle, right=1cm of gain4] (dot1) {};
  \node[lcircle, right=1cm of dot1] (output) {};

  %Rückführungen
  \node[wcircle, above= 2cm of gain3] (gain5) {$-\frac{1}{Q}$};

  \node[wcircle, below=2cm of gain4] (gain_vr2) {$V_r$};
    \node[sum, left=1cm of gain_vr2] (prod2) {$\prod$};
  \node[block, left=1cm of prod2] (gain6) {$\frac{1}{s}$};
  \node[wcircle, left=1cm of gain6] (gain7) {$-\frac{1}{RC}$};
  \node[wcircle, left=2.25cm of gain7] (gain8) {$-1$};

\node[lcircle] (in_vc) at (0,-1.5) {};
\node[bcircle] (dot2)  at (14.75,-1.5){};
  %Abzweigung
  %\coordinate[right=1cm of gain4] (afterW0);


  
  %\node[sum, right=of input] (sum2) {$\sum$};
  %\node[block, right=of sum1] (gain) {$K$};
  %\node[block, right=of gain] (system) {System};
  %\node[block, right=of system] (output) {Ausgang};

  % Connections
  \draw[connector] (input) -- (gain1) node[near start,above] {$V_i$};
  \draw[connector] (gain1) -- (sum1);
  \draw[connector] (sum1) -- (gain2)node[midway,above] {$V_4$};
  \draw[connector] (gain2) -- (sum2);
  \draw[connector] (sum2) -- (gain_vr1)node[midway,above] {$V_1$};
  \draw[connector] (gain_vr1) -- (prod1);
  \draw[connector] (prod1) -- (gain3);
  \draw[connector] (gain3) -- (gain4);
  %\draw[connector] (gain4) -- (dot1);
  \draw[connector] (gain4) -- (output) node[near end, above] {$V_2$};

  % Rückführung oben mit 90°-Biegung
  \draw[connector] (dot1) |- (gain5);
  \draw[connector] (gain5) -| (sum1.north);

  % Rückführung unten mit 90°-Biegung
  \draw[connector] (dot1) |- (gain_vr2);
  \draw[connector] (gain_vr2) -- (prod2);
  \draw[connector] (prod2) -- (gain6);
  \draw[connector] (gain6) -- (gain7);
  \draw[connector] (gain7) -- (gain8)node[midway, above]{$V_3$};
  \draw[connector] (gain8) -| (sum2.south) ;


\draw[connector] (in_vc) -| (prod1)node[near start, below,  xshift=-6.85cm]{$\frac{1}{V_c}$};
\draw[connector] (dot2) -| (prod2);


  %\draw[connector] (input) -- (sum1);
  %\draw[connector] (sum1) -- (gain);
  %\draw[connector] (gain) -- (system);
  %\draw[connector] (system) -- (output);

\end{tikzpicture}}
  \caption{Systemtheoretische Darstellung des \gls{vcf} ohne Phasendetektor (PD)}
  \label{fig:sb_sys_vcf}
\end{figure}


Die daraus hervorgehende Übertragungsfunktion lautet

\begin{equation}
\frac{V_2}{V_{in}} = \frac{-H_0 sRC \frac{V_c}{V_r}}{\left(sRC\frac{V_c}{V_r}\right)^{2}+\frac{sRC}{Q}\frac{V_c}{V_r}+1}
\end{equation}

Die Übertragungsfunktion des einfachen Biquads in der Standardform lautet

\begin{equation*}
\frac{V_2}{V_i} = -\frac{\frac{s}{\omega_0}H_0}{1 + \frac{s}{\omega_0 Q} + \frac{s^2}{\omega_0^2}}
\end{equation*}

mit $\omega_0 = \frac{1}{RC}$

Um nun auf die Gleichung für die Mittenfrequenz zu kommen muss die Übertragungsfunktion so normiert werden, dass der Nenner dem Nenner der Standardform entspricht. Bei Gleichsetzung der beiden höchsten Exponenten ergibt sich

\begin{equation*}
\left(\frac{s}{\omega_0}\right)^{2} = \left(sRC\frac{V_c}{V_r}\right)^{2}
\end{equation*}

Durch Herauskürzen von s, dem Exponenten und anschließender Termumformung nach $\omega_0$ ergibt sich 

\begin{equation}
    \omega_0 = \frac{V_r}{V_cRC}
    \label{eq:filtergrenzfrequenz}
\end{equation}


\textbf{Berechnung der Grenzfrequenz aus dem Video}\par
Laut ASLK-PRO Manual sollten $V_c$ und $V_r$ vertauscht sein. a die Mittenfrequenz jedoch eine physikalische Größe ist, die unabhängig von der Normierung sein sollte, könnte die im Manual vorgeschlagene Lösung entweder fehlerhaft sein oder auf einer anderen Normierungskonvention basieren (zweiteres ist unwahrscheinlicher, da der Rechenweg dadurch deutlich komplizierter zu sein scheint) YT vid \cite{YT_stf_lab5} sagt aber auch was anderes!!!\par
\medskip


\begin{figure}[!h]
  \centering
  \resizebox{0.8\textwidth}{!}{\input{sb_w0_herleitung.tex}}
  \caption{Vereinfachter Schaltplan zur Herleitung von $\omega_0$}
  \label{fig:bsb_w0}
\end{figure}

Laut Quelle (begleitendem YouTube Video zu diesem Experiment) \cite{YT_stf_lab5} kann die Formel für die Mittenfrequenz anhand dieser vereinfachten Schaltung abgeleitet werden. Die bekannte Formel für den Integrator lautet

\begin{equation}
  V_{out} = -\frac{V_{i}}{sCR} \label{eq:integrator}
\end{equation}

Da $V_{i}$ gleich dem Ausgang des Multiplizierers ist, ergibt sich für die Multiplizierergleichung

\begin{equation}
  V_{i} = \frac{V_{in} \cdot V_c}{V_r} \label{eq:multiplizierer}
\end{equation}

Wird \eqref{eq:multiplizierer} nun in \eqref{eq:integrator} eingesetzt, ergibt sich

\begin{equation*}
  V_{out} = -\frac{\frac{V_{in} \cdot V_c}{V_r}}{sCR} = -\frac{V_{in} \cdot V_c}{V_r \cdot sRC}
\end{equation*}

Um die Übertragungsfunktion zu erlangen, muss nun durch $V_{in}$ geteilt werden

\begin{equation*}
  \frac{V_{out}}{V_{in}} = -\frac{V_c}{V_r \cdot sRC} = - \frac{V_c}{V_r} \cdot \frac{1}{sRC}
\end{equation*}

Aus dem Term $\frac{1}{sRC}$ ergibt sich die Standardform $\omega_0 = \frac{1}{RC}$ mit einem zusätzlichen Faktor von
$\frac{V_c}{V_r}$, sodass sich die Mittenfrequenz wie folgt ergibt

\begin{equation}
  \omega_0 = \frac{V_c}{V_r \cdot RC} 
  \label{eq:freq-vcf}
\end{equation}

wobei

\begin{itemize}
\item $\omega_0$ die Durchlassfrequenz des Filters ist,
\item $V_c$ die Steuerspannung des \gls{vcf} ist,
\item $V_r$ der Referenzwert des Multiplizierers ist (laut Datenblatt: $V_r = \SI{10}{\volt}$),
\item $RC$ die Zeitkonstante des Filters beschreibt.
\end{itemize}


\subsection*{Herleitung der Funktionsweise des spannungsgesteuerten Filter}

\medskip
Im letzten Unterkapitel wurde die Beziehung zwischen der Phasendifferenz $\phi$ und der Steuerspannung $V_c$ hergeleitet. Nun ist von Interesse, welche Funktion diese Steuerpannung innerhalb des \gls{vcf} übernimmt und auf welche Weise die Phasendifferenz das Systemverhalten beeinflusst.\par
\medskip
Aus Abbildung~\ref{fig:op_out_freq} geht hervor, dass bei einer Phasenverschiebung zwischen \SI{90}{\degree} und \SI{270}{\degree} die DC-Ausgangsspannung des Multiplizierers negativ wird. Dadurch steigt die Steuerspannung $V_c$ nach der Integration mit positivem Vorzeichen an. Über die hergleitete Gleichung \ref{eq:filtergrenzfrequenz} lässt sich ein Zusammenhang zwischen $V_c$ und der Mittenfrequenz des Filters herstellen.\par
\medskip
Wird der ansteigende Wert für $V_c$ in Gleichung $\omega_0 = \frac{V_r}{V_cRC}$ eingesetzt, wird deutlich, dass die Zunahme von $V_c$ zu einer Verringerung von $\omega_0$ führt. Somit bewirken alle Phasenverschiebungen im Bereich von \SI{90}{\degree} bis \SI{270}{\degree} eine Verringerung der Mittenfrequenz. Für Phasendifferenzen in der Umgebung von \SI{0}{\degree} tritt der umgekerte Effekt ein: Das Potential von $V_c$ sinkt, wodurch die Mittenfrequenz des Systems ansteigt.\par 
\medskip
Um das resultierende Verhalten anschaulich zu untersuchen, wird der analytische Zusammenhang des Systems in Python umgesetzt. Die Phasendifferenz wird dabei nur am Nulldurchgang des internen Signal ermittelt um daraufhin die Frequenz des Signals um jeweils einen festen Betrag zu korregieren. Bei Betrachtung der Simulation in Abbildung~\ref{fig:phase2omega1} fällt auf, dass dieses vereinfachte System niemals eine Anpassung über nur ein paar Perioden erreicht, unabhängig von den Startwerten. Zudem scheint es, dass diese Anpassungen weder dafür sorgen, dass die Frequenz auf den Referenzwert konvergiert, noch dass sich die Phasenlage dem stabilen Arbeitspunkt von \SI{90}{\degree} annähert. \par




\begin{figure}[H]
    \centering
    \begin{tikzpicture}
        % Deine Farben direkt lokal definiert
        \definecolor{my_blue}{rgb}{0.039, 0.333, 0.549}
        \definecolor{my_orange}{rgb}{0.941, 0.471, 0.137}
        \definecolor{my_red}{rgb}{0.765, 0.02, 0.137}
        \definecolor{my_green}{rgb}{0, 0.569, 0.353}
        \definecolor{my_blue2}{rgb}{0.196, 0.706, 0.784}

        \begin{axis}[
            width=0.95\linewidth,
            height=0.45\linewidth,
            grid=both,
            grid style={dashed, gray!30},
            legend cell align=left,
            legend pos=south west,
            xlabel={Zeit / \si{\milli\second}},
            ylabel={Spannung / \si{\volt}},
            xmin=0, xmax=7,
            ymin=-1.2, ymax=1.4,
            % Verhindert das Verbinden über NaNs/Leerzeilen hinweg
            unbounded coords=jump,
            empty line=jump,
        ]
        \addplot[color=my_blue, thick, opacity=0.2, forget plot] 
                table[x=time, y=ref, col sep=comma]{Bilder/phase_ref_full.csv};

        \addplot[color=my_orange, line width=1.5pt, forget plot] 
                table[x=time, y=start, col sep=comma]{Bilder/phase_variation.csv};
        \addplot[color=my_red, line width=1.5pt] 
                table[x=time, y=down, col sep=comma]{Bilder/phase_variation.csv};
        \addlegendentry{Freq. decreases}
        \addplot[color=my_blue2, line width=1.5pt] 
                table[x=time, y=f1000, col sep=comma]{Bilder/phase_variation.csv};
        \addlegendentry{$f = 1000\,\si{\hertz}$}
        \addplot[color=my_green, line width=1.5pt] 
                table[x=time, y=up, col sep=comma]{Bilder/phase_variation.csv};
        \addlegendentry{Freq. increases}
        \end{axis}
    \end{tikzpicture}
    \caption{Signalverhalten bei diskreter Änderung der Frequenz anhand der Phasenlage}
    \label{fig:phase2omega1}
\end{figure}


%\begin{figure} [H]
%  \centering
%    \includegraphics[width=1\linewidth]{../Bilder/phase2omega1.png}
%    \caption{Signalverhalten bei diskreter Änderung der Frequenz anhand der Phasenlage}
%    \label{fig:phase2omega1}
%\end{figure}

Gut zu erkennen ist dieses Verhalten bei Betrachtung der ersten und dritten türkisen Halbwelle. Die Phasenlage verbessert
sich nicht im geringsten gegenüber der ersten Halbwelle. Das legt den Verdacht nahe, dass das System (sowohl Phasenlage
als auch Frequenz) ungedämpft schwingt. Um diesen Verdacht genauer zu überprüfen wird die weitere Programmierung
mithilfe eines KI-Sprachmodells entwickelt, wobei immer wieder darauf geachtet werden muss, dass alle Vorgaben
eingehalten werden und das System in der Realität auch so funktioniert.\par
\medskip




\begin{figure}[H]
    \centering
    \begin{tikzpicture}
        \begin{groupplot}[
            group style={
                group size=1 by 2,
                vertical sep=1.5cm,
                xlabels at=edge bottom, 
            },
            width=0.95\linewidth,
            height=0.22\textheight,
            grid=both,
            legend cell align=left,
            legend pos=south east,
            xlabel={Zeit / \si{\second}},
            tick label style={font=\footnotesize},
            title style={yshift=-1.5ex},
            every axis y label/.style={at={(ticklabel cs:0.5)}, rotate=90, anchor=near ticklabel},
            yticklabel style={text width=2.5em, align=right}, % Reserviert festen Platz
        ]
        \nextgroupplot[
            ylabel={Frequenz / \si{\hertz}},
            title={Frequenzverlauf},
            xmin=0, xmax=0.05,
        ]
            \addplot[color={rgb,255:red,10;green,85;blue,140}, solid, thick, line width=2]
                table[x=time, y=f_current, col sep=comma]{Bilder/regelung_freq.csv};
            \addlegendentry{$f_\text{current}$}
            \addplot[black, dashed, thick, line width=1.5]
                table[x=time, y=f_ref, col sep=comma]{Bilder/regelung_freq.csv};
            \addlegendentry{$f_\text{ref}=1000\,\si{\hertz}$}
            
        \nextgroupplot[
            ylabel={Phasendifferenz / \si{\degree}},
            title={Phasendifferenz},
            xmin=0, xmax=0.05,
        ]
            \addplot[color={rgb,255:red,0;green,145;blue,90}, mark=*, only marks, mark size=1.5pt, thick]
                table[x=time, y=phase_deg, col sep=comma]{Bilder/regelung_phase.csv};
            \addlegendentry{$\phi_\text{diff}$}
            \addplot[red, dashed, thick] coordinates {(0,-90)(0.05,-90)};
            \addplot[red, dashed, thick] coordinates {(0,90)(0.05,90)};
            \addlegendentry{$\pm90^\circ$-Grenze}
        \end{groupplot}
    \end{tikzpicture}
  \caption{Frequenz- und Phasendifferenzverlauf des Algoritmus mit Stufenregelung}
  \label{fig:phase2omega2}
\end{figure}


%\begin{figure} [h!]
%  \centering
%  \includegraphics[width=1\linewidth]{../Bilder/phase2omega2.png}
%  \caption{Frequenz- und Phasendifferenzverlauf des Algoritmus mit Stufenregelung}
%  \label{fig:phase2omega2}
%\end{figure}


Aus Abbildung~\ref{fig:phase2omega2} ist deutlich zu erkennen, dass sowohl die Frequenz als auch die Phasenlage ungedämpft schwingen. Dieses Verhalten lässt sich durch die Form der Phasenkennlinie des Systems (vgl.~Abbildung~\ref{fig:op_out_freq}) erklären. Die Steuerspannung reagiert besonders empfindlich auf Phasenänderungen im Bereich um \SI{180}{\degree}, da dort die Steigung der Kennlinie am größten ist. Schon kleine Abweichungen führen in diesem Bereich zu starken Änderungen der Mittenfrequenz, (was das System zusätzlich antreibt und damit die ungedämpfte Schwingung begünstigt.) \textbf{gefällt mir noch nicht}\par
\medskip
An den beiden Extremstellen der Kennlinie, bei \SI{90}{\degree} und \SI{270}{\degree}, ist die Steigung hingegen nahezu null. Diese Punkte bilden die beiden Gleichgewichtspunkte des Systems. Der Punkt bei \SI{90}{\degree} entspricht einem stabilen Arbeitspunkt, da kleine Phasenabweichungen nur geringe und ausgleichende Änderungen an der Mittenfrequenz bewirken. Der Punkt bei \SI{270}{\degree} stellt dagegen einen instabilen Arbeitspunkt dar: Bereits geringe Abweichungen führen zu Verstärkungen in die falsche Richtung, weshalb das System diesen Zustand nicht halten kann und sich davon wegbewegt.\par
\medskip
Um das Regelverhalten zu verbessern wird die Frequenz proportional zur Phasenänderung korregiert. Bei großen Phasenabweichungen im Bezug auf die Gleichgewichtspunkte folgt eine starke Korrektur der Frequenz, kleine Abweichungen führen zu kleinen Korrekturen. Diese proporionale Regelung wird ebenfalls in den Code übernommen. Zu Vereinfachungszwecken wird die Phasencharakteristik des \gls{pd} nicht Sinusförimg abgebildet, sondern nur angehähert stufenförmig. So erfolgt bei großer Phasendiffernz eine große Korrektur der Frequenz und bei kleiner eine geringe.\par



\begin{figure}[H]
    \centering
    \begin{tikzpicture}
        \begin{groupplot}[
            group style={
                group size=1 by 2,
                vertical sep=1.2cm,
                xlabels at=edge bottom,  % X-Label nur unten!
            },
            width=0.95\linewidth,
            height=0.22\textheight,
            grid=both,
            legend cell align=left,
            legend pos=south east,
            xlabel={Zeit / \si{\second}},
            tick label style={font=\footnotesize},
            title style={yshift=-1.5ex},
            every axis y label/.style={at={(ticklabel cs:0.5)}, rotate=90, anchor=near ticklabel},
            yticklabel style={text width=2.5em, align=right}, % Reserviert festen Platz
        ]
        \nextgroupplot[
            ylabel={Frequenz / \si{\hertz}},
            title={Frequenzverlauf},
            xmin=0, xmax=0.5,
        ]
            \addplot[color={rgb,255:red,10;green,85;blue,140}, solid, thick, line width=2]
                table[x=time, y=f_current, col sep=comma]{Bilder/p_regelung_freq.csv};
            \addlegendentry{$f_\text{current}$}
            \addplot[black, dashed, thick, line width=1.5]
                table[x=time, y=f_ref, col sep=comma]{Bilder/p_regelung_freq.csv};
            \addlegendentry{$f_\text{ref}=1000\,\si{\hertz}$}
            
        \nextgroupplot[
            ylabel={Phasendifferenz / \si{\degree}},
            title={Phasendifferenz},
            xmin=0, xmax=0.5,
        ]
            \addplot[color={rgb,255:red,0;green,145;blue,90}, mark=*, only marks, mark size=1pt, thick]
                table[x=time, y=phase_deg, col sep=comma]{Bilder/p_regelung_phase.csv};
            \addlegendentry{$\phi_\text{diff}$}
            \addplot[red, dashed, thick] coordinates {(0,90)(0.5,90)};
            \addplot[red, dashed, thick] coordinates {(0,-90)(0.5,-90)};
            \addlegendentry{$\pm90^\circ$-Grenze}
        \end{groupplot}
    \end{tikzpicture}
    \caption{Frequenz- und Phasendifferenzverlauf des Algoritmus mit proportionaler Regelung}
    \label{fig:phase2omega3}
\end{figure}


%\begin{figure}[H]
%    \centering
%    \includegraphics[width=1\linewidth]{../Bilder/phase2omega3.png}
%    \caption{Frequenz- und Phasendifferenzverlauf des Algoritmus mit proportionaler Regelung}
%    \label{fig:phase2omega3}
%\end{figure}


Leider ergibt sich durch das proportionale Frequenzanpassung nicht der gewünschte Dämpfungseffekt. Sogenannte PI-Regler haben in der Regelungstechnik häufiger dieses Problem. Zur Lösung diesen kann der Ausgang tiefpassgefiltert werden, danach könnte die Konvergierung zum gewollten Wert besser funktionieren. \cite{philippsenReg}\par
\medskip

\textbf{ich bekomme den algorithmus nicht mit dämpfung (TP) zum laufen. falls das noch passiert, hier die geschichte mit der Fallhöhe:}
Code wird nicht Stabil. Frequenz des Systems haut immer nach \SI{970}{\hertz} bzw.~\SI{1030}{\hertz} und weiter ab. 

Abbildung PIT Regler\par

(noch hypothese:)
kann es sein, dass die Einschwingzeit davon abhängig ist, wie groß die Phasen- und Frequenzdifferenz ist? 
Die Simulationsergebnisse zeigen zudem, dass die Einschwingzeit stark von der anfänglichen Phasen- und Frequenzabweichung abhängt: Je größer die Differenz zum Zielwert ist, desto länger dauert es, bis das System sich in die Nähe des stabilen Arbeitspunktes hineinbewegt.
\textbf{diese Überlegungen könnten auch gut als hypothese im mess bzw auswertungsteil rein, die mam dann wiederlegt oder bestätigt}

%\subsection{Ermittlung der Grenz-/ Mittenfrequenz bei unbekannten Parametern}
%nun in theorie teil 1 zusehen 


%\section{Einfluss des VCF auf die Filterfrequenz und praktische Parametersteuerung}

\section{Sensitivitätsanalyse von Filter und Detektor}
Was zeigt die Sensitivität im allgemeinen

Im Allgemeinen beschreibt die Sensitivität die Änderung einer Ausgangsgröße im Bezug auf die Änderung einer Eingangsgröße. Bei dem schon bekannten \gls{vco} beschreibt die Sensitivität beispielsweise, wie stark die Frequenz auf eine Änderung der Steuerspannung reagiert. Dabei vergrößert eine hohe Sensitivität die Tuningrange und Genauigkeit des Oszillators, jedoch wird dieser auch schwierger zu kontrollieren, da kleine Änderungen der Steuerspannung große Änderungen der Frequenz bewirken und Rauschen leichter eingefangen werden kann. Eine geringere Sensitivität verbessert die Stabilität des Systems. \par
%https://www.minicircuits.com/appdoc/AN95-005.html


\subsection{Sensitivität des Phasendetektors}
Die Sensitivität des \gls{pd} $K_{pd}$ kann durch folgende Gleichung beschrieben werden:

\begin{equation}
K_{pd} = \frac{dV_{av}}{d\phi} \quad in \si{\volt\per\radian}
\end{equation}


Wobei $V_{av}$ den durchschnittlichen Spannungswert des Ausgangs des Detektors $V_o$? \textbf{(Müsste $V_c$ sein!)} beschreibt.
Die Ableitung des Ausgangssignals im Duchschnitt nach der Phasendifferenz gibt an, wie stark sich die Ausgangsspannung bei Änderung der Phasendifferenz verändert. Für $\phi = \SI{90}{\degree}$ hat $V_{av}$ einen Wert von \SI{0}{\volt}.\cite{Lab_Kit_PRO}\par
\medskip
In Anlehnung an den Abhängigkeits- bzw. Sensitivitätsverlauf aus Abbildung~\ref{fig:op_out_freq} wird der Verlauf vereinfacht als Sinusfunktion dargestellt. Darauf wird die Funktion nach $\phi$ abgeleitet um die Sensitivität zu ermitteln.\par
\medskip



\begin{figure}[H]
    \centering
    \begin{tikzpicture}
        \begin{groupplot}[
            group style={group size=1 by 2, vertical sep=1.5cm},
            width=0.95\linewidth, height=0.35\linewidth,
            grid=both, grid style={dashed, gray!30},
            xmin=0, xmax=10,
            xtick={0, 2.5, 5, 7.5, 10},
            xticklabels={$0$, $\frac{\pi}{2}$, $\pi$, $\frac{3\pi}{2}$, $2\pi$},
            ylabel={Spannung / \si{\volt}},
            every axis y label/.style={at={(ticklabel cs:0.5)}, rotate=90, anchor=near ticklabel},
            yticklabel style={text width=2.5em, align=right}, % Reserviert festen Platz
        ]
        \nextgroupplot[title={Vereinfachung der Abhängigkeit zwischen Steuerspannung und Phasendifferenz}, legend pos=south east]
        \addplot[color={rgb,255:red,0;green,145;blue,90}, line width=1.5pt] 
                table[x=time, y=detec1100, col sep=comma]{Bilder/detec_freq_outputs.csv};
        \addplot[color={rgb,255:red,195;green,5;blue,35}, line width=1.5pt] 
                table[x=time_math, y=u_vereinfacht, col sep=comma]{Bilder/sensitivitaet_data.csv};
        \legend{Simulation, Vereinfacht}

        \nextgroupplot[title={Differenzierung der vereinfachten Abhängigkeit}, xlabel={Phase / \si{\radian}}]
        \addplot[color={rgb,255:red,195;green,5;blue,35}, line width=1.5pt] 
                table[x=time_math, y=u_diff, col sep=comma]{Bilder/sensitivitaet_data.csv};
        \end{groupplot}
    \end{tikzpicture}
    \caption{Vereinfachung und Differenzierung der Steuerspannungs-Phasendifferenz-Abhängigkeit}
    \label{fig:sens1}
\end{figure}


%\begin{figure} [H]
%    \centering
%    \includegraphics[width=1\linewidth]{../Bilder/sensitivität1.png}
%    \caption{xxx}
%    \label{fig:sens1}
%\end{figure}

Diese vereinfachte Simulation stimmt an den durch das Manual bekannten Punkten ($\SI{90}{\degree}$ und $\SI{270}{\degree}$) überein. Die Amplitude der Funktion ist allerdings deutlich größer als erwartet. Mit einem Maximalwert von \SI{2000}{\volt} würde die Regelung sehr früh um den Arbeitspunkt stagnieren, da die Verstärker nur bis zu \SI{15}{\volt} ausgeben können.\par
\medskip
Diese Amplitude ergibt sich in diesem Fall durch die Ableitung der hohen Frequenz. In diesem Fall ändert die Frequenz also die Sensitivität des \gls{pd}. eine geringere Frequenz führt über die Ableitung zu einer geringeren Amplitude und somit zu einer geringeren Sensitivität. \par


\textbf{Kann man hier noch irgendwas rechnen? z.B. für eine Bestimmte frequenz? oder ist das immer 20V/pi wie in der Abbildung weiter oben?}
was sagt die sensitivität in diesem Fall aus? kann dazu noch etwas gemessen/ simuliert werden?

\subsection{Sensitivität des spannungsgesteuerten Filter}
\textbf{Achtung: Dieser Teil wird über die im Buch beschribene Formel der Mittenfrequenz bestimmt}

Bei Ableitung der Gleichung für die Mittenfrequenz \eqref{eq:freq-vcf} nach der Steuerspannung $V_c$ ist zu erkennen, wie empfindlich die Filterfrequenz auf die anliegende Steuerspannung reagiert.

\begin{equation*}
\frac{d \omega_0}{d V_c} = \frac{1}{V_r \cdot RC}
\end{equation*}

Durch einfaches Umstellen der selben Gleichung \eqref{eq:freq-vcf} ergibt sich

\begin{equation*}
\frac{\omega_0}{V_c} = \frac{1}{V_r \cdot RC}
\end{equation*}

So ergibt sich ein Gesamtzusammenhang, der die Empfindlichkeit der Filterfrequenz gegenüber der Änderung der Steuerspannung beschreibt.

\begin{equation}
\frac{d \omega_0}{d V_c} =  \frac{\omega_0}{V_c} \label{eq:empf_w0-Vc}
\end{equation}

Die Größen der Mittenfrequenz und der Steuerspannung verhalten sich direkt Proportional zu einander. So entspricht die relative Änderung der Frequenz der relativen Änderung der Steuerspannung. Mit anderen Worten: Verdoppelt sich die Steuerspannung verdoppelt sich auch die Mittenfrequenz. (bei linearer Abhänigkeit)\par
\medskip


Die Sensitivität des gesammten \gls{vcf} lässt sich durch folgende Gleichung beschreiben

\begin{equation}
\frac{d \phi}{d V_c} = \frac{d \phi}{d \omega_0} \cdot \frac{d \omega_0}{d V_c} \label{eq:ges-sensitivität}
\end{equation}

Hierbei zeigt diese Gleichung, wie stark die Phasendifferenz auf eine Änderung der Steuerspannung reagiert (unter berücksichtigung der Sensitivität des Filters und des \gls{pd}).

Der hintere Teil der Gleichung wird in \eqref{eq:empf_w0-Vc} beschrieben. Nun muss nur noch $\frac{d\phi}{d\omega}$ ermittelt werden.
\medskip

Dafür kann eine Übertragungsfunktion des Filters verwendet werden. Hierbei bietet sich die Tiefpass-Übertragungsfunktion an, da diese einen Phasengang zeigt, der seinen Startwert bei $0^\circ$ hat. 

\begin{align}
  H(s) &= \frac{V_{o_{TP}}}{V_i} = \frac{H_0}{1 + \frac{s}{\omega_0 Q} + \frac{s^2}{\omega_0^2}} \label{eq:u_tp} \\
  H(s) &= H(j \omega_r) = \frac{H_0}{1 + \frac{j \omega_r}{\omega_0 Q} + \frac{(j \omega_r)^2}{\omega_0^2}} = \frac{H_0}{1 - \frac{ \omega_r^2}{\omega_0^2} + j\frac{\omega_r}{\omega_0 Q} } \nonumber
\end{align}

Der Phasenwinkel einer Übertragungsfunktion wird berechnet, indem Zähler und Nenner jeweils als komplexe Zahlen betrachtet werden und für beide die Argumente ermittelt werden, also der Winkel ihrer komplexen Werte im Frequenzbereich. Der Phasenwinkel der gesuchten Übertragungsfunktion ergibt sich durch

\begin{equation}
  \phi = \arg(Zähler) - \arg(Nenner)
\end{equation}

wobei $\arg(z)$ der Winkel der komplexen Zahl $z$ ist. Für diese Übertragungsfunktion ergibt sich also ein $\phi$ von

\begin{equation}
  \phi = -tan^{-1} \left(\frac{\frac{\omega_r}{\omega_0 Q}}{1- (\frac{\omega_r}{\omega_0}) ^2}\right)
\end{equation}

Da der Zähler der Übertragungsfunktion $0^\circ$ hat fällt dieser aus der Rechnung heraus. 

\textbf{Hinweis}: im Manual steht im Nenner der tan funktion nur ein $\omega_0$ ohne quadrat! Zudem fehlt das - Zeichen.
\medskip

$\omega_r$ ist die Eingangskreisfrequenz. Die gesammte Formel beschreibt die Phasenverschiebung des Filters zum Eingangssignal.

\medskip

An dieser Stelle könnte nun eine Lange rechnung stehen wie man zu diesem Ergebniss kommt. Hier die Kurzfassung

\begin{equation}
  \frac{d \phi}{d \omega_0}= -\frac{2Q}{\omega_0}
\end{equation}

Eingesetzt in die Gleichung \eqref{eq:ges-sensitivität} ergibt sich daraus die Sensitivität:

\begin{equation}
  \frac{d \phi}{d V_c} = \frac{d \phi}{d \omega_0} \cdot \frac{d \omega_0}{d V_c} = - \frac{2Q}{\omega_0} \cdot \frac{\omega_0}{V_c}=-\frac{2Q}{V_c}
\end{equation}


\textbf{schreiben was das genau besagt, damit man das vielleicht später in einer messung validieren kann}

Im enteffekt beschreibt die Sensitivität des self-Tuned Filters also, wie stark sich die Phasenabweichung $\phi$ bei varierung der Steuerspannung $V_c$ ändert. Dabei ist zu sehen, dass die Sensitivität direkt Porportional zur Güte $Q$ des Filters ist. Je höer die Güte, desto empfindlicher reagiert das System auf die Eingangssteuergröße.


\section{Begrenzungen des Self-Tuning-Bereichs eines aktiven Filters}

\subsection{Bestimmung der maximalen Mittenfrequenz eines aktiven Filters}

Wie zuvor schon besprochen lässt sich die bauteilbedingte Mittenfrequenz eines aktiven Filters über die Gleichung $\omega_0 = \frac{1}{RC}$ beschreiben. Durch Variation der Werte von $R$ und $C$ lässt sich diese Frequenz in der Theorie beliebig verändern. In der Praxis können parasitäre Kapazitäten sowie weitere nichtideale Bauteileigenschaften das Filterverhalten beeinflussen, besonders wenn Standardbauteile mit größeren Toleranzen verwendet werden.\par
\medskip
Auch die Wahl des verwendeten \gls{opv} spielt für die maximal erreichbare Mittenfrequenz eine wichtige Rolle. So sind vor allem die Parameter für das Verstärkungs-Bandbreite-Produkt (eng. Gain-Bandwidth-Product, GBW) und die Slew-Rate (SR) entscheidend. Das GBW gibt an, bis zu welcher Frequenz der OpAmp eine Verstärkung von 1 stabil liefern kann. Die Slew-Rate beschreibt die maximale Anstiegsrate der Ausgangsspannung (maximale Änderungsrate der Ausgangsspannung) des OpAmps.\cite{halbleiter_4Quad}\par
\medskip
Zusätzlich beeinflusst die gewählte Filtertopologie die maximal erreichbare  Mittenfrequenz. Höhere Filterordnungen oder Kaskaden mehrerer Stufen beanspruchen jeweils einen Teil der verfügbaren Verstärkungsbandbreite, sodass die Mittenfrequenz insgesamt weiter sinkt.\par
\medskip
In der Praxis wird zunächst die maximal benötigte Mitten- bzw. Grenzfrequenz des Filters bestimmt. Anschließend wird ein \gls{opv} ausgewählt, dessen GBW im Regelfall mindestens um den Faktor \numrange{10}{100} höher liegt als die angestrebte Grenzfrequenz. Darauf basierend werden die Werte für $R$ und $C$ so dimensioniert, dass die gewünschte Mitten/Grenzfrequenz erreicht wird.\cite{microchipADN003}, \cite{YT_opv_active_filters}\par
\medskip
Hier kann einmal auf die werte der eigenen Bauteile eingegangen werden und was das für das gesammtsystem bedeutet. z.b. (aus gedanken) der OP hat ein GBW von \SI{4}{\mega\hertz}, der Multiplizierer von \SI{10}{\mega\hertz}, also bestimmt der OP die maximal verwendbare Frequenz. diese sollte nach daumenregel von oben bei ca ... liegen. Um das System dahingehend zu verbessern kann nun ein anderer \gls{opv} herausgesucht werden. (Wird viellehct in Version \num{2} so gemacht)

%https://dk.farnell.com/enhancing-signal-quality-understanding-the-role-of-filters-trc-ar
(Farnell beschreibt auch unterschied aktiv und passiv filter)

\subsection{Bestimmung des maximalen Tune-Bereich des hier verwendeten Filter}

Eine weitere interessante Frage ist, über welchen Bereich die Mittenfrequenz des Filters durch die self-Tune-Funktion verstellt werden kann, ohne die physischen Bauelemente zu verändern. (Zudem ist wichtig herauszufinden wie ich das Messen kann)\par
\medskip
Die Mindestanforderung an den Tune-Bereich ist, die entstehenden Bauteiltolleranzen auszugleichen. Standardbauteile besitzen meist eine Toleranz von etwa \SI{5}{\percent}. Da sowohl $R$ also auch $C$ diese Abweichungen besitzen können, kann die Mittenfrequenz $\omega_0$ allein schon durch diese um bis zu \SI{10}{\percent} variieren. Der designte Self-Tuned Filter sollte also mindestens diesen Bereich vollständig abdecken können. \par
\medskip
Der Tune-Berich sollte also bei mindestens $\pm\SI{10}{\percent}$ um die Bauteilbedingte Mittenfrequenz liegen. Andernfalls wäre der Filter nicht in der Lage, allein die üblichen Fertigungs- und Bauteiltoleranzen auszugleichen.\par
\medskip
In der Literatur finden sich nur wenige direkte Angaben zum real erreichbaren Tune-Bereich self-tuned aktiver Filter. Daher sollte dies besser durch eine Simulation oder Messung ermittelt werden. \par

%https://www.digitalxplore.org/up_proc/pdf/143-143133720956-59.pdf


\section{Frequenzdetektion des Eingangssignals}

Wie im oberen Teil beschrieben, kann ein Self-Tuned Filter präzise auf die Frequenz des empfangenden Signals abgestimmt werden. Daher bietet es sich an, die eingehende Frequenz über die \gls{mcu}  auszuwerten, um ohne Messgeräte wie dem RedPitaya die eingestellte Mittenfrequenz des Filters zu bestimmen. Dabei wird davon ausgegangen dass das self-tuning funktioniert und der Filter die Eingangsfrequenz als Mittenfrequenz angenommen hat.\par
\medskip
Die Umsetzung der Frequenzmessung kann analog oder digital erfolgen. Als  analoge Option könnte ein Frequenz-Spannungs-Wandler (F/V-Converter) verwendet werden, der die Frequenz des Eingangssignals in eine proportionale Gleichspannung umwandelt. Diese Spannung kann anschließend über einen \gls{adc} an der \gls{mcu} ausgelesen werden. Der große Vorteil liegt hierbei in der schnellen Reaktionszeit. Nachteilig ist, dass das Eingangssignal vorverarbeitet werden muss, um einem Rechtecksignal zu entsprechen. Zudem ist der Dynamikbereich begrenzt, da sehr hohe oder niedrige Frequenzen spezielle Anpassungen erfordern. Für die Realisierung dieses Verfahrens wären daher mehrere externe Komponenten notwendig. Diese führen nach der Installation zu Einschränkungen in der Flexibilität, da sie nicht mehr so leicht verändert werden können.\par
\medskip
Demgegenüber spricht der digitale Ansatz dafür, dass deutlich weniger externe Bauteile erforderlich sind und durch die Softwareprogrammierung eine höhere Flexibilität gegeben ist. So kann die Frequenz beispielsweise mit einem Nulldurchgangszähler ermittelt werden. Dieser zählt die Anzahl der Nulldurchgänge oder Pulse pro Sekunde und teilt diesen Wert durch 2, sodass die Frequenz in Hertz ermittelt werden kann. Der begrenzende Faktor ist hierbei die Abtastrate der \gls{mcu}. Da die \gls{mcu} Rechteck- bzw Tacktsignale erwartet, müssen analoge Signale wie Sinus, Dreieck und Sägezahn zuvor ebenfalls vorverarbeitet werden. Dafür eignet sich beispielsweise ein Komparator, der die analogen Signale in saubere Rechteckimpulse umwandelt und auch bei kleinen Pegeln zuverlässig arbeitet. Eine zusätzliche Vorverarbeitung kann auch bei diesem Verfahren dazu verwendet werden, stabilere Messergebnisse zu erhalten. \par
\medskip
Ein verbleibendes Problem besteht darin, dass diese Methode keine Mischsignale mit mehreren Frequenzanteilen analysieren kann. In solchen Fällen bietet sich die Verwendung einer schnellen Fourier-Transformation (FFT) an, um das Spektrum des Eingangssignals auszuwerten und die einzelnen Frequenzkomponenten zu identifizieren.\par
\medskip
Brauche ich das Überhaupt? kommen überhaupt mischsignale auf mein system drauf? Wie verhält sich das system wenn mischsignale auf dieses gebracht werden?

\end{document}
