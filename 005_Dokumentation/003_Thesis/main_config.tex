\documentclass[a4paper,12pt]{book}

% --- Sprache, Kodierung & Schrift ---
\usepackage[T1]{fontenc}          % korrekte Silbentrennung & Umlaute
\usepackage[utf8]{inputenc}       % Eingabekodierung UTF-8
\usepackage[ngerman]{babel}       % deutsche Spracheinstellungen
\usepackage{microtype}            % bessere Typografie (Abstände, Randausgleich)
\usepackage{csquotes}             % korrekte Anführungszeichen mit babel (Pflicht für biblatex!)

% --- Mathematik ---
\usepackage{amsmath, amssymb, mathtools} % erweiterte Matheumgebungen
\usepackage{siunitx}                     % SI-Einheiten und Zahlenformatierung
\sisetup{
  locale = DE,                           % Komma als Dezimaltrennzeichen
  per-mode = symbol,
  detect-all                            % übernimmt Schriftart aus Umgebung
}

% --- Layout ---
\usepackage{geometry}
\geometry{
  top=2cm,
  bottom=3cm,
  left=2.5cm,
  right=2.5cm
}
\setlength\parindent{0pt} % kein Absatzeinzug
\setlength\parskip{0.5em} % etwas Abstand zwischen Absätzen

% --- Grafiken ---
\usepackage{graphicx}        
\usepackage{float}
\usepackage{caption}
\usepackage{subcaption}
\usepackage{booktabs,tabularx}
\usepackage{longtable}
\usepackage{array, makecell}
\usepackage{enumitem}
\usepackage{tikz}
\usepackage{circuitikz}
\usetikzlibrary{arrows.meta, positioning}

% --- Kopf- und Fußzeilen ---
\usepackage{fancyhdr}
\setlength\headheight{26pt}
\setlength\headsep{35pt}
\pagestyle{fancy}
\fancyhf{}
\lhead{Nils Renner (5197659)}
\chead{Vorbereitung Bachelorthesis}
\rhead{\includegraphics[width=4cm]{Bilder/Logo_HSB_Hochschule_Bremen.png}}

% --- Codeblöcke ---
\usepackage{listings}
\usepackage{xcolor}
\lstdefinestyle{arduino}{
    language=C,
    basicstyle=\ttfamily\footnotesize,
    keywordstyle=\color{blue},
    commentstyle=\color{gray},
    stringstyle=\color{red},
    numbers=left,
    numberstyle=\tiny\color{gray},
    stepnumber=1,
    breaklines=true,
    frame=single,
    captionpos=b
}

% --- Literatur ---
\usepackage[backend=biber, style=ieee]{biblatex}
\addbibresource{Literaturverzeichnis.bib}

% --- Sonstiges ---
\usepackage{lastpage}  % für "Seite X von Y"
